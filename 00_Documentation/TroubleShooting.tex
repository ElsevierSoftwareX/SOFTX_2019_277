\newpage
\part{Error messages and Troubleshooting} \label{part:troub}
\section{Error and Warning messages}
Most errors occur when the wrong python interpreter is used or when rasters or layouts have bad formats or when the information stated in the input file (see Sec.~\ref{sec:inpfile}) is erroneous. The package writes process errors and descriptions to logfiles. When the GUI encounters problems, it directly provides causes and remedies in pop-up infoboxes. The common error and warning messages, which can be particularly raised by the package (alphabetical order) are listed in the following with detailed descriptions of causes and remedies. Most error messages are written to the logfiles, but some exception errors are only printed to the terminal because they occur before logging could even be started. Such non-logged \texttt{ExceptionErrors} are listed at the bottom of Sec.~\ref{sec:errors}. Some non-identifiable errors raised by the \texttt{arcpy} package disappear after rebooting the system.

\subsection{Error messages}\label{sec:errors}
\begin{itemize}
	
	\item[$\triangleright$] \textbf{\texttt{ERROR 000641: Too few records for analysis.}}\\		
	\begin{itemize}
		\item[\textit{Cause}\hspace{0.27cm}] This \texttt{arcpy} error message occurs here when \pythoninline{arcpy.CalculateAreas_stats} tries to compute the area of an empty shapefile.
		\item[\textit{Remedy}] If this error occurs within the calculation of AUA (annually usable habitat area) calculations, it may be ignored because some discharges do not provide any usable habitat area for a target fish species within a defined project area.\\
		Otherwise, trace back files and check the shapefile consistency.\\
	\end{itemize}
	
	\item[$\triangleright$] \textbf{\texttt{ERROR 999998: Unexpected Error.}}\\
This is an operating system error and it can indicate different error conditions, i.e., the real reasons may have various error sources. Some of the most probable causes are:
	\begin{itemize}
		\item[\textit{Cause}\hspace{0.27cm}] Usage of the wrong python interpreter
		\item[\textit{Remedy}] -- Make sure to use the \texttt{ArcGIS\textbf{x64}XX.X} python interpreter (64 bit).\\
							   -- Make sure that all input rasters are in \textit{(Esri) Grid} format and well placed in the folder \texttt{LifespanDe sign/Input/\textit{condition}/}.\\
							   -- Rebooting the system can help in some cases.\\
	\end{itemize}
	
	\item[$\triangleright$] \textbf{\texttt{ERROR: .cache folder in use.}}
	\begin{itemize}
		\item[\textit{Cause}\hspace{0.27cm}] The content in the cache folder is blocked by another software and the output is probably affected.
		\item[\textit{Remedy}] Close the software that blocks \texttt{.cache}, including \texttt{explorer.exe}, other instances of \texttt{python} or \texttt{ArcGIS} and rerun the code. Also re-logging may be required if the folder cannot be unlocked.\\
	\end{itemize}
	
	\item[$\triangleright$] \textbf{\texttt{ERROR: .cache folder will be removed by package controls.}}
	\begin{itemize}
		\item[\textit{Cause}\hspace{0.27cm}] \pythoninline{arcpy} could not clean up the \texttt{.cache} folder and the task is passed to \textit{Python}'s \pythoninline{os} package. The content in the cache folder is blocked by another process and the output is probably affected.
		\item[\textit{Remedy}] Close the software that blocks \texttt{.cache}, including \texttt{explorer.exe}, other instances of \textit{Python} or \textit{ArcGIS} and rerun the code. Also re-logging may be required if the folder cannot be unlocked.\\
	\end{itemize}
	
	\item[$\triangleright$]\textbf{\texttt{ERROR: (arcpy) in \textit{PAR}.}}
	\begin{itemize}
		\item[\textit{Cause}\hspace{0.27cm}] Similar to \pythoninline{ExceptionERROR: (arcpy) ...}. The error is raised by the \pythoninline{analysis_...}, \pythoninline{design_...} and other functions when \pythoninline{arcpy} raster calculations could not be performed. Missing rasters, bad raster assignments, errors in input geodata files, or bad raster calculation expressions are possible reasons. The error can also occur when the \texttt{Spatial} license is not available.
		\item[\textit{Remedy}] See \pythoninline{ExceptionERROR: (arcpy) ...}\\
	\end{itemize}
	
	\item[$\triangleright$]\textbf{\texttt{ERROR: Analysis stopped ([...] failed).}}
	\begin{itemize}
		\item[\textit{Cause}\hspace{0.27cm}] Raised by \pythoninline{analysis(...)} function in \texttt{LifespanDesign/feature{\myUnderscore}analysis.py} when it encountered an error.
		\item[\textit{Remedy}] Trace back the error message in brackets. If a results raster could not be saved, it means that the analyzed feature has no application, i.e., the results raster is empty, and therefore, it cannot be saved.\\
	\end{itemize}
	
	\item[$\triangleright$]\textbf{\texttt{ERROR: Area calculation failed.}}
	\begin{itemize}
		\item[\textit{Cause}\hspace{0.27cm}] Raised by \pythoninline{calculate_wua(self)} of \textit{HabitatEvaluation}'s \pythoninline{CHSI()} class in \texttt{HabitatEvaluation/cHSI.py} when it could not calculate the usable habitat area (see Sec.~\ref{sec:heauamethods}).
		\item[\textit{Remedy}] -- Ensure that the AUA threshold has a meaningful value between 0.0 and 1.0 (Sec.~\ref{sec:heintro}).\\
							 -- Ensure that neither the directory \texttt{HabitatEvaluation/.cache/} nor the directory \texttt{HabitatEvalu ation/AUA/} or their contents are in use by other program.\\
							 -- Review the input settings according to Sec.~\ref{sec:hequick}.\\
							 -- Follow up earlier error messages.\\
	\end{itemize}
	
	\item[$\triangleright$]\textbf{\texttt{ERROR: Bad assignment of x/y values in coordinate input file.}}
	\begin{itemize}
		\item[\textit{Cause}\hspace{0.27cm}] Raised by the \pythoninline{coordinates_read(self)} function of the \pythoninline{Info()} class in either \texttt{LifespanDesign/cRead InpLifespan.py} or \texttt{MaxLifespan/cReadActionInput.py} when \texttt{mapping.inp} has bad assignments of $x$-$y$ coordinates.
		\item[\textit{Remedy}] Ensure that the coordinate definitions in \texttt{mapping.inp} (\texttt{LifespanDesign/.templates/} or \texttt{Act ionPlanner/.templates/}) correspond to the definitions in Sec.~\ref{sec:inpmaps}.\\
	\end{itemize}
	
	\item[$\triangleright$]\textbf{\texttt{ERROR: Bad call of map centre coordinates. Creating squared-x layouts.}}
	\begin{itemize}
		\item[\textit{Cause}\hspace{0.27cm}] Raised by \pythoninline{get_map_extent(self, direction)} function of the \pythoninline{Info()} class in either \texttt{LifespanDesign/cReadInpLifespan.py} or \texttt{MaxLifespan/cReadActionInput.py} when \texttt{mapping.inp} has bad assignments of $x$-$y$ coordinates.
		\item[\textit{Remedy}] -- \textit{LifespanDesign}: Ensure that the file \texttt{mapping.inp} exists in the directory \texttt{LifespanDesign/.templates/} corresponding to the definitions in Sec.~\ref{sec:inpmaps}.\\
													 -- \textit{MaxLifespan}: Ensure that the file \texttt{mapping.inp} exists in the directory \texttt{MaxLifespan/.temp lates/} corresponding to the definitions in Sec.~\ref{sec:inpmaps}.\\
													 -- General: Replace \texttt{mapping.inp} with the original file and re-apply modifications strictly following Sec.~\ref{sec:inpmaps}.
	\end{itemize}
	
	\item[$\triangleright$]\textbf{\texttt{ERROR: Bad mapping input file.}}
	\begin{itemize}
		\item[\textit{Cause}\hspace{0.27cm}] Raised by either \pythoninline{get_map_extent(self, direction)}, \pythoninline{coordinates_read(self)} or \pythoninline{get_map_scale(self)} function of the \pythoninline{Info()} class in either \texttt{LifespanDesign/cReadInpLifespan.py} or \texttt{MaxLifespan/cReadActionInput.py} when \texttt{mapping.inp} has wrong formats or it is missing.
		\item[\textit{Remedy}] See \texttt{ERROR: Bad call of map centre coordinates [...].}\\
	\end{itemize}
	
	\item[$\triangleright$]\textbf{\texttt{ERROR: Boundary shapefile in arcpy.PolygonToRaster[...].}}
	\begin{itemize}
		\item[\textit{Cause}\hspace{0.27cm}] Raised the HabitatEvaluation module's \pythoninline{make_boundary_ras(self, shapefile)} function (\texttt{cHSI.py}) when it could not convert a provided shapefile defining calculation boundaries to a raster and load it as \pythoninline{arcpy.Raster(.../HabitatEvaluation/HSI/condition/bound_ras)}.
		\item[\textit{Remedy}] Verify that a the selected boundary shapefile (Sec.~\ref{sec:hebound}) has a valid rectangle and an \texttt{Id} field value of \texttt{1} for that rectangle.\\
	\end{itemize}
	
	\item[$\triangleright$]\textbf{\texttt{ERROR: Boundary shapefile provided but [...].}}
	\begin{itemize}
		\item[\textit{Cause}\hspace{0.27cm}] Raised the HabitatEvaluation module's \pythoninline{make_chsi(self, fish, boundary_shp)} function (\texttt{cHSI.py}) when the "To Raster" conversion of the provided shapefile defining calculation boundaries failed.
		\item[\textit{Remedy}] See \texttt{ERROR: Boundary shapefile in arcpy.PolygonToRaster[...].}\\
	\end{itemize}
	
	\item[$\triangleright$]\textbf{\texttt{ERROR: Calculation of cell statistics failed.}}
	\begin{itemize}
		\item[\textit{Cause}\hspace{0.27cm}] Raised by \pythoninline{identify_best_features(self)} of \textit{MaxLifespan}'s \pythoninline{ArcPyContainer()} class in \texttt{MaxLifes span/cActionAssessment.py} when \pythoninline{arcpy.sa.CellStatistics()} could not be executed.
		\item[\textit{Remedy}] -- The latest feature added to the internal best lifespan raster may contain inconsistent data. Manually load the last feature raster (the logfile tells the feature name) into \textit{ArcMap} and trace back the error. If needed, re-run lifespan/design Raster Maker.\\
							-- In the case that the error occurs already with the first feature added, the \textit{MaxLifespan}'s \texttt{zero} raster may be corrupted. The remedy described for the error message \texttt{ExceptionERROR: Unable to create ZERO Raster. Manual intervention required} can be used to manually re-create the \texttt{zero} raster.\\
	\end{itemize}
	
	\item[$\triangleright$]\textbf{\texttt{ERROR: Calculation of volume from RASTER failed.}}
	\begin{itemize}
		\item[\textit{Cause}\hspace{0.27cm}] The \pythoninline{volume_computation(self)} function of the \pythoninline{ModifyTerrain()} class in \texttt{ModifyTerrain/cModify Terrain.py} raises this error when the command \pythoninline{arcpy.SurfaceVolume_3d(RASTER, "", "ABOVE", 0.0, 1.0)} failed.
		\item[\textit{Remedy}] -- Ensure that an \textit{ArcGIS} \texttt{3D} extension license is available.\\
							   -- Ensure that manually modified (Customary Feature) raster DEMs contain valid data.\\
							   -- Ensure that the input directory of manually modified (Customary Feature) raster DEMs is correct (default: \texttt{ModifyTerrain/Input/DEM/\textit{condition}/}).\\
	\end{itemize}
	
	\item[$\triangleright$]\textbf{\texttt{ERROR: Cannot find FEAT max. lifespan raster.}}
	\begin{itemize}
		\item[\textit{Cause}\hspace{0.27cm}] The automated terrain modification with grading and/or widen features uses max. lifespan rasters (maps) to identify relevant areas. If the \pythoninline{get_action_raster(self, feature_name)} function of the \pythoninline{ModifyTerrain()} class in \texttt{ModifyTerrain/cModifyTerrain.py} cannot find max. lifespan rasters in the defined max. lifespan raster directory (default: \texttt{MaxLifespan/Output/Rasters/\textit{condition}/}), it raises this error message.
		\item[\textit{Remedy}] Ensure that grading and/or widen max. lifespan rasters exist in the defined input folder (default \texttt{MaxLifespan/Output/Rasters/\textit{condition}/}) and that the names of the rasters contain the feature shortname, i.e., \texttt{grade} and/or \texttt{widen}.\\
	\end{itemize}
	
	\item[$\triangleright$]\textbf{\texttt{ERROR: Cannot find flow depth raster.}}
	\begin{itemize}
		\item[\textit{Cause}\hspace{0.27cm}] Raised by \pythoninline{make_chsi(self, fish, boundary_shp)} of the \textit{HabitatEvaluation}'s \pythoninline{CHSI()} class in \texttt{Habitat Evaluation/cHSI.py} when it could associate a flow depth raster based on the name of a habitat suitability index (HSI) raster name.
		\item[\textit{Remedy}] -- Ensure that the flow depth raster names in \texttt{RiverArchitect/01{\myUnderscore}Conditions/\textit{condition}/} strictly comply with the naming conventions described in Sec.~\ref{sec:input}.\\
							   -- Ensure that the HSI rasters are stored in \texttt{.../HabitatEvaluation/HSI/\textit{condition}/}, with the correct raster names including information about the discharge (see Sec.~\ref{sec:heoutputhhsi}).\\
	\end{itemize}
	
	\item[$\triangleright$]\textbf{\texttt{ERROR: Cannot find modified DEM. Ensure that file names contain 'dem'.}}
	\begin{itemize}
		\item[\textit{Cause}\hspace{0.27cm}] The volume difference calculation and mapping of Custom CAD-modified DEM rasters failed because the \pythoninline{get_cad_rasters_for_volume(self, feat_id)} function of the \pythoninline{ModifyTerrain()} class in \texttt{ModifyTerrain/cModifyTerrain.py} cannot find the raster files.
		\item[\textit{Remedy}] Ensure that Custom CAD-modified DEM rasters exist in the defined input folder (default \texttt{ModifyTerrain/Input/DEM/\textit{condition}/}) and that the names of the rasters contain the keyword \texttt{dem}, e.g., a valid raster name is \texttt{dem14{\myUnderscore}mod}, or feature shortname, i.e., \texttt{cust}.\\
	\end{itemize}
	
		\item[$\triangleright$]\textbf{\texttt{ERROR: Could not access Fish.xlsx (...).}}
	\begin{itemize}
		\item[\textit{Cause}\hspace{0.27cm}] The \pythoninline{get_hsi_curve(self, species, lifestage, par)} function of the \pythoninline{Fish()} class (\texttt{HabitatEvaluatio}\\\texttt{n/cFish.py}) or the \pythoninline{main()} function in \texttt{s40{\myUnderscore}compare{\myUnderscore}wua.py} raise this error message when it cannot access \texttt{Fish.xlsx} or copy read values from the \texttt{/HabitatEvaluation/AUA/\textit{condition}} directory.
		\item[\textit{Remedy}] Ensure that neither \texttt{HabitatEvaluation/.templates/Fish.xlsx} nor any file in \texttt{/HabitatEva luation/AUA/\textit{condition}} is used by another program.\\
	\end{itemize}
	
	\item[$\triangleright$]\textbf{\texttt{ERROR: Could not add cover HSI.}}
	\begin{itemize}
		\item[\textit{Cause}\hspace{0.27cm}] The \pythoninline{make_chsi(self, fish)} function of the \pythoninline{CHSI()} class (\texttt{HabitatEvluation/cHSI.py}) raises this error message when it failed to add cover HSI rasters.
		\item[\textit{Remedy}] Manually verify cover HSI rasters in \texttt{HabitatEvaluation / HSI/} and recompile cover HSI rasters if needed (see Sec.~\ref{sec:hemakecovhsi}).\\
	\end{itemize}		
	
	\item[$\triangleright$]\textbf{\texttt{ERROR: Could not append PDF page XX to map assembly.}}
	\begin{itemize}
		\item[\textit{Cause}\hspace{0.27cm}] The \pythoninline{make_pdf_maps(self, *args)} or \pythoninline{map_custom(self, input_ras_dir, *args)}, \pythoninline{map_reach(self,}\\
	\pythoninline{ reach_id, feature_id, *args)} functions of the \pythoninline{Mapper} class in \texttt{MaxLifespan/cMapper.py} or\\ \texttt{ModifyTerrain/cMapModifiedTerrain.py} raise this error when they failed to map the current page (extent).
		\item[\textit{Remedy}] -- \textit{MaxLifespan}: Ensure that the definitions of \texttt{MaxLifespan/.templates/mapping.inp} are correct, analog to the descriptions of the \textit{LifespanDesign} module in Sec.~\ref{sec:inpmaps}.\\
								 -- \textit{ModifyTerrain}: Also refer to error message \texttt{ERROR: Could not create PDF}.\\
								 -- General: Ensure that no other program accesses the \texttt{MaxLifespan/.cache/}, \texttt{ModifyTerrain/.cache/} or \texttt{MaxLifespan/Output/}, \texttt{ModifyTerrain/Output/} directories or its contents.\\
	\end{itemize}
	
	\item[$\triangleright$]\textbf{\texttt{ERROR: Could not calculate CellStatistics (raster comparison).}}
	\begin{itemize}
		\item[\textit{Cause}\hspace{0.27cm}] Raised by \pythoninline{compare_raster_set(self, ...)} function of the \pythoninline{ArcPyAnalysis()} class in \texttt{LifespanDesign/cLifespanDesignAnalysis.py} when the provided it failed to combine the lifespan according to the provided input rasters (hydraulic or scour fill or morphological units).
		\item[\textit{Remedy}] Manually open the input rasters and ensure that they comply with the requirements stated in Sec.~\ref{sec:input}.\\
	\end{itemize}
	
	\item[$\triangleright$]\textbf{\texttt{ERROR: Could not create PDF}}
	\begin{itemize}
		\item[\textit{Cause}\hspace{0.27cm}] The \pythoninline{map_custom(self, input_ras_dir, *args)} function of the \pythoninline{Mapper()} class (\texttt{ModifyTerrain/cMap ModifiedTerrain.py}) raises this error message when it \pythoninline{arcpy.mapping.ExportToPDF(self.mxd, self.output_map_dir + map_name, image_compression="ADAPTIVE", resolution=96)} failed.
		\item[\textit{Remedy}] Ensure consistent layout template definitions according to Sec.~\ref{sec:mtlyt}.\\
	\end{itemize}
	
	\item[$\triangleright$]\textbf{\texttt{ERROR: Could not create Raster of the project area.}}
	\begin{itemize}
		\item[\textit{Cause}\hspace{0.27cm}] Raised by \pythoninline{set_project_area(self)} of \textit{ProjectMakers}'s \pythoninline{CAUA()} class in \texttt{ProjectMaker/cWUA.py} when it failed to convert the project area shapefile to a raster, which it needs for limiting spatial calculations to the project extent.
		\item[\textit{Remedy}] Ensure that the project was correctly delineated (Sec.~\ref{sec:pminp2}).\\
	\end{itemize}
	
	\item[$\triangleright$]\textbf{\texttt{ERROR: Could not crop raster to defined flow depth.}}
	\begin{itemize}
		\item[\textit{Cause}\hspace{0.27cm}] The \pythoninline{crop_input_raster(self, fish_species, fish_lifestage, depth_raster_path)} function of the \pythoninline{CovHSI(HHSI)} class (\texttt{HybitatEvluation/cHSI.py}) raises this error message when it failed cropping the raster with the spatial analyst operation \pythoninline{Con((Float(h_raster) >= h_min), cover_type_raster)}.
		\item[\textit{Remedy}] Ensure that the provided flow depth file (selected in the GUI) contains valid data and that \texttt{Fish.xlsx} contains a minimum flow depth value for the selected fish species and lifestage.\\
	\end{itemize}
	
	\item[$\triangleright$]\textbf{\texttt{ERROR: Could not export PDF page no. XX}}
	\begin{itemize}
		\item[\textit{Cause}\hspace{0.27cm}] The \pythoninline{make_pdf_maps(self, *args)} function of the \pythoninline{Mapper} class in \texttt{MaxLifespan/cMapper.py} raises this error when \texttt{MaxLifespan/.templates/mapping.inp} contains invalid xy-coordinates (format).
		\item[\textit{Remedy}] Ensure the definitions of \texttt{MaxLifespan/.templates/mapping.inp} analog to the descriptions of the \textit{LifespanDesign} module in Sec.~\ref{sec:inpmaps}.\\
	\end{itemize}
	
	\item[$\triangleright$]\textbf{\texttt{ERROR: Could not find max. lifespan Rasters.}}
	\begin{itemize}
		\item[\textit{Cause}\hspace{0.27cm}] Error raised by the \pythoninline{main()} function in \texttt{ProjectMaker/s20{\myUnderscore}plantings{\myUnderscore}delineation.py}) when the defined directory of max. lifespan rasters contains invalid or corrupted raster data.
		\item[\textit{Remedy}] -- Ensure the correct usage of variables and input definitions (Sec.~\ref{sec:pmquick}).\\
		-- Ensure that max. lifespan Rasters were generated without errors; if necessary, visually control the consistency of max. lifespan rasters in \texttt{\ldots{}/MaxLifespan/Products/Rasters/\emph{condition}{\myUnderscore}\emph{reach}{\myUnderscore}l\\yr20{\myUnderscore}plants/} and \texttt{\ldots{}/MaxLifespan/Products/}\texttt{Rasters/\emph{condition}{\myUnderscore}\emph{reach}{\myUnderscore}lyr20{\myUnderscore}plants/} or \ldots{}\texttt{bioengineering}(cf. Sec.~\ref{sec:pmactm}).\\
	\end{itemize}
	
	\item[$\triangleright$]\textbf{\texttt{ERROR: Could not find any worksheet.}}
	\begin{itemize}
		\item[\textit{Cause}\hspace{0.27cm}] Error raised by the \pythoninline{open_wb(self)} function of the \pythoninline{Read()} class in \texttt{ProjectMaker/cIO.py}) when the concerned workbook contains errors.
		\item[\textit{Remedy}] -- Ensure the correct usage of \texttt{HabitatEvaluation/.templates/Fish.xlsx} (Sec.~\ref{sec:hefish}).\\
		-- Ensure the correct adaptation of \texttt{ProjectMaker/.../REACH{\myUnderscore}stn{\myUnderscore}assessment{\myUnderscore}vii.xlsx}\\    (Sec.~\ref{sec:pmcq}).\\
	\end{itemize}
	
	\item[$\triangleright$]\textbf{\texttt{ERROR: Could not find sheet.}}
	\begin{itemize}
		\item[\textit{Cause}\hspace{0.27cm}] Error raised by the \pythoninline{open_wb(self)} function of the \pythoninline{Read()} class in \texttt{HabitatEvaluation/cHabitatIO.py}) when the template workbook contains errors.
		\item[\textit{Remedy}] Ensure the correct usage of \texttt{HabitatEvaluation/.templates/Fish.xlsx} (Sec.~\ref{sec:hefish}) and the completeness of \texttt{HabitatEvaluation/.templates/Q{\myUnderscore}def{\myUnderscore}hab{\myUnderscore}template{\myUnderscore}si.xlsx} and \texttt{Ha bitatEvaluation/.templates/Q{\myUnderscore}def{\myUnderscore}hab{\myUnderscore}template{\myUnderscore}us.xlsx}. If either template workbook is corrupted or does not exist, re-install missing files.\\
	\end{itemize}
	
	\item[$\triangleright$]\textbf{\texttt{ERROR: Could not find sheet ``extents'' in computation{\myUnderscore}extents.xlsx.}}
	\begin{itemize}
		\item[\textit{Cause}\hspace{0.27cm}] Error raised by the \pythoninline{get_reach_coordinates(self, internal_reach_id)} function of the \pythoninline{Read()} class in \texttt{.site{\myUnderscore}packages/riverpy/cTerrainIO.py}) when the \texttt{extents} sheet in the reach coordinate spreadsheet (\texttt{ModifyTerrain/.templates/computation{\myUnderscore}extents.xlsx}) could not be read.
		\item[\textit{Remedy}] Ensure the correct setup of \texttt{ModifyTerrain/.templates/computation{\myUnderscore}extents.xlsx}\\   (Sec.~\ref{sec:mtsetreaches}).\\
	\end{itemize}
	
	\item[$\triangleright$]\textbf{\texttt{ERROR: Could not find the cover input geofile [...]}}
	\begin{itemize}
		\item[\textit{Cause}\hspace{0.27cm}] Error raised by the \pythoninline{__init__(self, ...)} function of the \pythoninline{CovHSI(HHSI)} class in \texttt{HabitatEvaluation/cHSI.py}) when the input cover geofile could not be read or is missing.
		\item[\textit{Remedy}] Ensure that a geofile (raster or shapefile) exists in the specified \texttt{\textit{condition}} folder for the specified cover type (checkbox activated in the GUI). The \texttt{Help} button in the GUI provides more information on required geofiles and Sec.~\ref{sec:heprin}.\\
	\end{itemize}
	
	\item[$\triangleright$]\textbf{\texttt{ERROR: Could not interpolate exceedance probability of Q =  [...]}}
	\begin{itemize}
		\item[\textit{Cause}\hspace{0.27cm}] Raised by \pythoninline{interpolate_flow_exceedance(self, Q_value)} of \textit{HabitatEvaluation}'s \pythoninline{FlowAssessment()} class in \texttt{HabitatEvaluation/cHSI.py} when the flow duration curve contains invalid data.
		\item[\textit{Remedy}] Ensure the correct setup of the used flow duration curve in \texttt{HabitatEvaluation/FlowDuration Curves/}. The file structure must correspond to that of the provided template \texttt{flow{\myUnderscore}duration{\myUnderscore}templa te.xlsx} and all discharge values need to be positive floats. Review Sec. \ref{sec:hemakehsi} for details.\\
	\end{itemize}
	
	\item[$\triangleright$]\textbf{\texttt{ERROR: Could not open workbook.}}
	\begin{itemize}
		\item[\textit{Cause}\hspace{0.27cm}] Error raised by the \pythoninline{__init__(self)} function of the \pythoninline{Read()} class in \texttt{ProjectMaker/cIO.py}) when the concerned workbook contains errors.
		\item[\textit{Remedy}] Ensure the correct usage of the concerned workbook (Part~\ref{part:pm}.\\
	\end{itemize}
	
	\item[$\triangleright$]\textbf{\texttt{ERROR: Could not load newly created Raster of the project area.}}
	\begin{itemize}
		\item[\textit{Cause}\hspace{0.27cm}] Raised by \pythoninline{set_project_area(self)} of \textit{ProjectMakers}'s \pythoninline{CAUA()} class in \texttt{ProjectMaker/cWUA.py} when the converted the project area shapefile is corrupted.
		\item[\textit{Remedy}] Ensure that the project was correctly delineated (Sec.~\ref{sec:pminp2}).\\
	\end{itemize}
	
	\item[$\triangleright$]\textbf{\texttt{ERROR: Could not perform spatial radius operations [...].}}
	\begin{itemize}
		\item[\textit{Cause}\hspace{0.27cm}] The \pythoninline{spatial_join_analysis(self, rater, curve_data)} function of the \pythoninline{CovHSI(HHSI)} class (\texttt{Habitat Evaluation/cHSI.py}) raises this error message when one or several spatial calculations failed, including \pythoninline{arcpy.RasterToPoint_conversion[...]}, \pythoninline{arcpy.SpatialJoin_analysis[...]} and / or \\ \pythoninline{arcpy.PointToRaster_conversion[...]}.
		\item[\textit{Remedy}] Ensure that the cover input files and habitat suitability (curve) parameters are properly defined according to Sec.~\ref{sec:hemakecovhsi}.\\
	\end{itemize}
	
	\item[$\triangleright$]\textbf{\texttt{ERROR: Could not process information from [...].}}
	\begin{itemize}
		\item[\textit{Cause}\hspace{0.27cm}] The \pythoninline{main()} function in \texttt{ProjectMaker/s40{\myUnderscore}compare{\myUnderscore}wua.py} raises this error message when it could not calculate the annually usable habitat area for condition or (set of) discharge(s).
		\item[\textit{Remedy}] Ensure that the variable (parameters) are properly defined according to Sec.~\ref{sec:pmquick} and that the \textit{HabitatEvaluation} module contains the required information.\\
	\end{itemize}
		
	\item[$\triangleright$]\textbf{\texttt{ERROR: Could not read parameter type [...] from Fish.xlsx.}}
	\begin{itemize}
		\item[\textit{Cause}\hspace{0.27cm}] The \pythoninline{get_hsi_curve(self, species, lifestage, par)} function of the \pythoninline{Fish()} class (\texttt{HabitatEvaluati on/cFish.py}) raises this error message when it cannot read a habtiat suitability curve from \texttt{Fish.xlsx}.
		\item[\textit{Remedy}] -- Ensure that\texttt{HabitatEvaluation/.templates/Fish.xlsx} is not opened in any other program.\\
							   -- Ensure that a habitat suitability curve is defined in {Fish.xlsx} for the considered hydraulic or cover parameter according to Sec.~\ref{sec:hefish}.\\
	\end{itemize}	
	
	\item[$\triangleright$]\textbf{\texttt{ERROR: Could not retrieve reach coordinates.}}
	\begin{itemize}
		\item[\textit{Cause}\hspace{0.27cm}] The automated terrain modification with grading and/or widen features in the \pythoninline{modification_manager(self, feat_id)} function of the \pythoninline{ModifyTerrain()} class in \texttt{ModifyTerrain/cModifyTerrain.py} raises this error when the reach extents defined in \texttt{ModifyTerrain/.templates/computation{\myUnderscore}ex tents.xlsx} are not readable. In particular, the command \pythoninline{self.reader.get_reach_coordinates(self.reaches.dict_id_int_id[self.current_reach_id])} caused the error.
		\item[\textit{Remedy}] -- Follow the instructions in Sec.~\ref{sec:mtsetreaches} for correct reach definitions.\\
								-- If the \textit{ModifyTerrain} module is externally loaded, ensure the correct definition of features and feature shortnames (see Sec.~\ref{sec:mtaltrun}).\\
	\end{itemize}
	
	\item[$\triangleright$]\textbf{\texttt{ERROR: Could not run AUA analysis.}}
	\begin{itemize}
		\item[\textit{Cause}\hspace{0.27cm}] The \pythoninline{main()} function in \texttt{ProjectMaker/s40{\myUnderscore}compare{\myUnderscore}wua.py} raises this error message when it could not calculate AUA.
		\item[\textit{Remedy}] Trace back warning and other error messages. Ensure the correct definition of parameters, creation of required geodata, and file naming (Part~\ref{part:pm})\\
	\end{itemize}
	
	\item[$\triangleright$]\textbf{\texttt{ERROR: Could not save best lifespan raster.}}
	\begin{itemize}
		\item[\textit{Cause}\hspace{0.27cm}] Raised by \pythoninline{identify_best_features(self)} of \textit{MaxLifespan}'s \pythoninline{ArcPyContainer()} class in \texttt{MaxLifespan/cActionAssessment.py} when the calculated internal best lifespan raster is corrupted.
		\item[\textit{Remedy}] -- Check prior WARNING and ERROR messages.\\
								-- Ensure that neither the directory \texttt{MaxLifespan/.cache/} nor the directory \texttt{MaxLifespan/Output/} or their contents are in use by other programs.\\
	\end{itemize}
	
	\item[$\triangleright$]\textbf{\texttt{ERROR: Could not save CSI raster associated with ...}}
	\begin{itemize}
		\item[\textit{Cause}\hspace{0.27cm}] Raised by \pythoninline{make_chsi_hydraulic(self, fish)} of \textit{HabitatEvaluation}'s \pythoninline{CHSI()} class in \texttt{HabitatEvalua tion/cHSI.py} when the calculated cHSI raster is empty or corrupted.
		\item[\textit{Remedy}] -- Ensure that neither the directory \texttt{HabitatEvaluation/.cache/} nor the directory \texttt{HabitatEval uation/AUA/} or their contents are used by another program.\\
							 -- Review the input settings according to Sec.~\ref{sec:hequick}.\\
	\end{itemize}
	
	\item[$\triangleright$]\textbf{\texttt{ERROR: Could not save cover / H HSI [...] raster ...}}
	\begin{itemize}
		\item[\textit{Cause}\hspace{0.27cm}] Raised by \pythoninline{make_hhsi(self, fish_applied)} of \textit{HabitatEvaluation}'s \pythoninline{HHSI()} class in \texttt{HabitatEvalua tion/cHSI.py} when the calculated HHSI raster is empty or corrupted.
		\item[\textit{Remedy}]  -- Ensure that no other software uses data from neither the \texttt{HabitatEvaluation/} nor the \texttt{Stream Restoration/01{\myUnderscore}Conditions/} directories.\\
							 -- Review the input flow velocity and depth rasters according to Sec.~\ref{sec:input}.\\
	\end{itemize}
	
	\item[$\triangleright$]\textbf{\texttt{ERROR: Could not save WORKBOOK.}}
	\begin{itemize}
		\item[\textit{Cause}\hspace{0.27cm}] The \pythoninline{main()} function in \texttt{ProjectMaker/s40{\myUnderscore}compare{\myUnderscore}wua.py} raises this error message when it could not save \texttt{AUA{\myUnderscore}evaluation\textit{{\myUnderscore}unit}.xlsx}.
		\item[\textit{Remedy}] Ensure that the workbook exists, has valid contents, and is not opened by another program.\\
	\end{itemize}
		
	\item[$\triangleright$]\textbf{\texttt{ERROR: Could not save AUA-CHSI raster.}}
	\begin{itemize}
		\item[\textit{Cause}\hspace{0.27cm}] Raised by \pythoninline{calculate_wua(self)} of \textit{HabitatEvaluation}'s \pythoninline{CHSI()} class in \texttt{HabitatEvaluation/cHSI.py} when the calculated cHSI raster is empty or corrupted.
		\item[\textit{Remedy}] -- Ensure that the AUA threshold has a meaningful value between 0.0 and 1.0 (Sec.~\ref{sec:heintro}).\\
							 -- Ensure that neither the directory \texttt{HabitatEvaluation/.cache/} nor the directory \texttt{HabitatEval uation/AUA/} or their contents are in use by other programs.\\
							 -- Review the input settings according to Sec.~\ref{sec:hequick}.\\
	\end{itemize}
	
	\item[$\triangleright$]\textbf{\texttt{ERROR: Could not load existing Raster of the project area.}}
	\begin{itemize}
		\item[\textit{Cause}\hspace{0.27cm}] Raised by \pythoninline{set_project_area(self)} of \textit{ProjectMakers}'s \pythoninline{CAUA()} class in \texttt{ProjectMaker/cWUA.py} when it found a raster that delineates the project area, but this raster is corrupted. The function requires the shapefile to raster conversion to limit applicable rasters to the project extent range, which is done with raster calculator operations.
		\item[\textit{Remedy}] -- Ensure that the project was correctly delineated (Sec.~\ref{sec:pminp2}).\\
							 -- Manually inspect the project delineation raster.\\
	\end{itemize}
	
	\item[$\triangleright$]\textbf{\texttt{ERROR: Could not transfer net AUA gain.}}
	\begin{itemize}
		\item[\textit{Cause}\hspace{0.27cm}] The \pythoninline{main()} function in \texttt{ProjectMaker/s40{\myUnderscore}compare{\myUnderscore}wua.py} raises this error message when it could not copy the calculated AUA from \texttt{AUA{\myUnderscore}evaluatio\\\textit{{\myUnderscore}unit}.xlsx} to \texttt{\emph{REACH{\myUnderscore}stn{\myUnderscore}costs{\myUnderscore}vii.xlsx}}.
		\item[\textit{Remedy}] Open \texttt{AUA{\myUnderscore}evaluation\textit{{\myUnderscore}template{\myUnderscore}unit}.xlsx} and verify the calculated values. Trace back potential error sources in the CHSI rasters \texttt{/HabitatEvaluation/} folder and other error messages.\\
	\end{itemize}	
	
	\item[$\triangleright$]\textbf{\texttt{ERROR: Could not transfer AUA data for [FISH].}}
	\begin{itemize}
		\item[\textit{Cause}\hspace{0.27cm}] The \pythoninline{main()} function in \texttt{ProjectMaker/s40{\myUnderscore}compare{\myUnderscore}wua.py} raises this error message when it could not retrieve AUA data from the \texttt{/HabitatEvaluation/AUA/} module to  \texttt{AUA{\myUnderscore}evaluation\\\textit{{\myUnderscore}unit}.xlsx}.
		\item[\textit{Remedy}] Open \texttt{AUA{\myUnderscore}evaluation\textit{{\myUnderscore}template{\myUnderscore}unit}.xlsx} and verify the calculated values. Trace back potential error sources in the CHSI rasters \texttt{/HabitatEvaluation/} folder and other error messages.\\
	\end{itemize}	
	
	\item[$\triangleright$]\textbf{\texttt{ERROR: Could not write value to CELL [...]}}
	\begin{itemize}
		\item[\textit{Cause}\hspace{0.27cm}] Error raised by the \pythoninline{write_data_cekk(self, column, row, value)} function of the \pythoninline{Write()} class in \texttt{Habi tatEvaluation/cHabitatIO.py}) when it cannot write a value to \texttt{RiverArchitect/Habit atEvaluation/AUA/\textit{condition{\myUnderscore}fill}.xlsx}.
		\item[\textit{Remedy}] Close all applications that may use \texttt{RiverArchitect/HabitatEvaluation/AUA/\textit{condition}{\myUnderscore}\\\textit{fill}.xlsx}. Detailed information on \textit{HabitatEvaluation} workbook outputs are available in Sec.~\ref{sec:hemakehsi}.\\
	\end{itemize}
	
	\item[$\triangleright$]\textbf{\texttt{ERROR: Could not write AUA data for [FISH].}}
	\begin{itemize}
		\item[\textit{Cause}\hspace{0.27cm}] The \pythoninline{main()} function in \texttt{ProjectMaker/s40{\myUnderscore}compare{\myUnderscore}wua.py} raises this error message when it could not write the calculated AUA to when it cannot write a value to \texttt{AUA{\myUnderscore}evaluation\textit{{\myUnderscore}template{\myUnderscore}}\\\texttt{unit}.xlsx}.
		\item[\textit{Remedy}] Ensure that the workbook is not opened by another program and / or visually verify that the concerned \texttt{CHSI} rasters contain valid values.\\
	\end{itemize}
	
	\item[$\triangleright$]\textbf{\texttt{ERROR: Cover raster calculation (check input data).}}
	\begin{itemize}
		\item[\textit{Cause}\hspace{0.27cm}] Raised by \pythoninline{call_analysis(self, curve_data)} of \textit{HabitatEvaluation}'s \pythoninline{CovHSI(HHSI)} class in \texttt{HabitatEv aluation/cHSI.py} when the cover HSI raster calculation failed.
		\item[\textit{Remedy}]  Ensure that the input geofiles (raster or shapefile) are correctly set up according to  Sec.~\ref{sec:hemakecovhsi}~ff.\\
	\end{itemize}	
	
	\item[$\triangleright$]\textbf{\texttt{ERROR: Extent is not FLOAT. Substituting to extent = 7000.00.}}
	\begin{itemize}
		\item[\textit{Cause}\hspace{0.27cm}] Raised by the \pythoninline{save_design(self, name)} or \pythoninline{save_lifespan(self, name)} functions of the \pythoninline{ArcPyAnalysis} class in either \texttt{LifespanDesign/cLifespanDesignAnalysis.py} when the output folder for rasters (the folder directory is stated in the logfile) contains rasters of the same name which cannot be deleted.
		\item[\textit{Remedy}] Ensure that no other program uses the raster output folder and consider moving existing files in that folder to \texttt{LifespanDesign/Products/Rasters/\textit{condition}}.\\
	\end{itemize}
	
	\item[$\triangleright$]\textbf{\texttt{ERROR: Existing files are locked. Consider deleting [...] file structure.}}
	\begin{itemize}
		\item[\textit{Cause}\hspace{0.27cm}] Raised by the \pythoninline{get_map_extent(self, direction)} function of the \pythoninline{Info()} class in either \texttt{LifespanDesign/cReadInpLifespan.py} or \texttt{MaxLifespan/cReadActionInput.py} when \texttt{mapping.inp} has bad assignments of $x$-$y$ coordinates (not a number).
		\item[\textit{Remedy}] See \texttt{ERROR: Bad call of map centre coordinates ... }\\
	\end{itemize}
	
	\item[$\triangleright$]\textbf{\texttt{ERROR: Failed calling \textit{PAR} analysis of \textit{FEATURE}.}}
	\begin{itemize}
		\item[\textit{Cause}\hspace{0.27cm}] Special case of \textbf{\texttt{ERROR: Function analysis}}, which may occur after code modifications.
		\item[\textit{Remedy}] -- Make sure that the \pythoninline{self.parameter_list}s of features (Sec.~\ref{sec:add-feat}) has valid entries that also occur in \pythoninline{analysis_call(*args)} (\texttt{LifespanDesign/feature{\myUnderscore}analysis.py}).\\
								 -- Make sure that valid function names exist in \texttt{LifespanDesign/cLifespanDesignAnalysis.py} (Sec.~\ref{sec:add-ana}).\\
	\end{itemize}
	
	\item[$\triangleright$]\textbf{\texttt{ERROR: Failed to access computation{\myUnderscore}extents.xlsx.}}
	\begin{itemize}
		\item[\textit{Cause}\hspace{0.27cm}] Error raised by the \pythoninline{get_reach_coordinates(self, internal_rach_id)} function of the \pythoninline{Read()} class in \texttt{Mod ifyTerrain/cReadTerrainIO.py}) when the reach coordinate spreadsheet (\texttt{ModifyTerrain/.templates/computation{\myUnderscore}extents.xlsx}) could not be read.
		\item[\textit{Remedy}] Ensure correct setup of \texttt{ModifyTerrain/.templates/computation{\myUnderscore}extents.xlsx} (Sec. \ref{sec:mtsetreaches}).\\
	\end{itemize}
	
	\item[$\triangleright$]\textbf{\texttt{ERROR: Failed to access /load  Fish.xlsx / Q{\myUnderscore}def{\myUnderscore}hab ...}}
	\begin{itemize}
		\item[\textit{Cause}\hspace{0.27cm}] Error raised by the \pythoninline{open_wb(self)} and \pythoninline{make_condition_xlsx(self, fish_sn)} functions of the \pythoninline{Read()} class in \texttt{HabitatEvaluation/cHabitatIO.py}) when the template workbook contains errors.
		\item[\textit{Remedy}] Ensure the correct usage of \texttt{HabitatEvaluation/.templates/Fish.xlsx} (Sec.~\ref{sec:hefish}) and the completeness of \texttt{HabitatEvaluation/.templates/Q{\myUnderscore}def{\myUnderscore}hab{\myUnderscore}template{\myUnderscore}si.xlsx} and \texttt{Ha bitatEvaluation/.templates/Q{\myUnderscore}def{\myUnderscore}hab{\myUnderscore}template{\myUnderscore}us.xlsx}. If either template workbook is corrupted or does not exist, re-install missing files.\\
	\end{itemize}
	
	\item[$\triangleright$]\textbf{\texttt{ERROR: Failed to access WORKBOOK.}}
	\begin{itemize}
		\item[\textit{Cause}\hspace{0.27cm}] Error raised by the \pythoninline{write_volumes(self, ...)} function of the \pythoninline{Writer()} class in \texttt{.site{\myUnderscore}packages/riverpy/cTerrainIO.py}) or the \pythoninline{__init__(..)} function of \textit{ProjectMakers}'s \pythoninline{Read()} class in \texttt{Project Maker/cIO.py} when the \pythoninline{WORKBOOK} is inaccessible or locked by another program.
		\item[\textit{Remedy}] Ensure that the concerned workbook exists and no other program uses the workbook.\\
	\end{itemize}
	
	\item[$\triangleright$]\textbf{\texttt{ERROR: Failed to add raster.}}
	\begin{itemize}
		\item[\textit{Cause}\hspace{0.27cm}] Raised by \pythoninline{read_hyd_rasters(self)} of \textit{HabitatEvaluation}'s \pythoninline{HHSI()} class in \texttt{HabitatEvaluation/cHSI.py} when is could not find hydraulic input rasters.
		\item[\textit{Remedy}] -- Ensure that no other software uses data from neither the \texttt{HabitatEvaluation/} nor the \texttt{Stream Restoration/01{\myUnderscore}Conditions/} directories.\\
							 -- Review the input flow velocity and depth rasters according to Sec.~\ref{sec:input}.\\
	\end{itemize}
	
	\item[$\triangleright$]\textbf{\texttt{ERROR: Failed to create WORKBOOK.}}
	\begin{itemize}
		\item[\textit{Cause}\hspace{0.27cm}] Error raised by the \pythoninline{write_volumes(self, ...)} function of the \pythoninline{Writer()} class in \texttt{.site{\myUnderscore}packages/riverpy/cTerrainIO.py}) when the \texttt{template} it could not add new sheets in \texttt{ModifyTerrain/Output/Spreadsheets/\textit{condition}{\myUnderscore}volumes.xlsx} or write to copies of \texttt{ModifyTerrain/Output/Spreadsheets/volume{\myUnderscore}template.xlsx}.
		\item[\textit{Remedy}] Trace back earlier error messages, ensure that no other program locked \texttt{ModifyTerrain/Output/Spreadsheets/\textit{condition}{\myUnderscore}volumes.xlsx} and ensure that \texttt{ModifyTerrain/Output/Spr eadsheets/volume{\myUnderscore}template.xlsx} was not deleted.\\
	\end{itemize}
	
	\item[$\triangleright$]\textbf{\texttt{ERROR: Failed to open Fish.xlsx. Ensure that the workbook is not open.}}
	\begin{itemize}
		\item[\textit{Cause}\hspace{0.27cm}] Raised by the \pythoninline{edit_xlsx(self)} function of the \pythoninline{Fish()} class in \texttt{HabitatEvaluation/cFish.py} when \\ \texttt{HabitatEvaluation/.templates/Fish.xlsx} is opened by another program or non-existent.
		\item[\textit{Remedy}] Ensure that the file \texttt{HabitatEvaluation/.templates/Fish.xlsx} exists and close any software that may use the workbook.\\
	\end{itemize}
	
	\item[$\triangleright$]\textbf{\texttt{ERROR: Failed to read coordinates from computation{\myUnderscore}extents.xlsx (return 0).}}
	\begin{itemize}
		\item[\textit{Cause}\hspace{0.27cm}] Error raised by the \pythoninline{get_reach_coordinates(self, internal_rach_id)} function of the \pythoninline{Read()} class in \texttt{.site{\myUnderscore}packages/riverpy/cTerrainIO.py}) when the reach coordinate spreadsheet (\texttt{ModifyTer rain/.templates/computation{\myUnderscore}extents.xlsx}) contains invalid data.
		\item[\textit{Remedy}] Ensure correct setup of \texttt{ModifyTerrain/.templates/computation{\myUnderscore}extents.xlsx} (Sec. \ref{sec:mtsetreaches}).\\
	\end{itemize}
	
	\item[$\triangleright$]\textbf{\texttt{ERROR: Failed to read maximum depth to water value for [...].}}
	\begin{itemize}
		\item[\textit{Cause}\hspace{0.27cm}] Error raised by the \pythoninline{lower_dem_for_plants} function of the \pythoninline{ModifyTerrain} class in \texttt{ModifyTerrain/cModifyTerrain.py}) when the threshold workbook (\texttt{LifespanDesign/.templates/thresh old{\myUnderscore}values.xlsx}) is not accessible or does not contain values for \textit{Depth to groundwater (min) / max}  contains invalid data.
		\item[\textit{Remedy}] Ensure the correct setup of \texttt{LifespanDesign/.templates/threshold{\myUnderscore}values.xlsx} (Sec.~\ref{sec:modthresh}). Note that \pythoninline{ModifyTerrain} starts reading depth to ground water values column by column, until it meets a non-numeric value.\\
	\end{itemize}
	
	\item[$\triangleright$]\textbf{\texttt{ERROR: Failed to save PDF map assembly.}}
	\begin{itemize}
		\item[\textit{Cause}\hspace{0.27cm}] The \pythoninline{make_pdf_maps(self, *args)} function of the \pythoninline{Mapper} class in \texttt{MaxLifespan/cMapper.py} or \texttt{ModifyTerrain/cMapper.py} raises this error when the map assembly is corrupted.
		\item[\textit{Remedy}] Ensure that no other program accesses the \texttt{MaxLifespan/.cache/}, \texttt{ModifyTerrain/.cache/} or \texttt{MaxLifespan/Output/}, \texttt{ModifyTerrain/Output/} directories or their contents.\\
	\end{itemize}
	
	\item[$\triangleright$]\textbf{\texttt{ERROR: Failed to save WORKBOOK.}}
	\begin{itemize}
		\item[\textit{Cause}\hspace{0.27cm}] Raised by \pythoninline{calculate_wua(self)} of \textit{HabitatEvaluation}'s \pythoninline{CHSI()} class in \texttt{HabitatEvaluation/cHSI.py} when it could not save \texttt{\textit{condition}{\myUnderscore}\textit{fill}.xlsx}.
		\item[\textit{Remedy}] Ensure that no other software uses \texttt{HabitatEvaluation/AUA/\texttt{\textit{condition}{\myUnderscore}\textit{fill}.xlsx}}.\\
	\end{itemize}
	
	\item[$\triangleright$]\textbf{\texttt{ERROR: Failed to set reach extents -- output is corrupted.}}
	\begin{itemize}
		\item[\textit{Cause}\hspace{0.27cm}] The automated terrain modification with grading and/or widen features in the \pythoninline{lower_dem_for_plants(self, feat_id, extents)} function of the \pythoninline{ModifyTerrain()} class in \texttt{ModifyTerrain/cModifyTerrain.py} raises this error when the reach extents defined in \texttt{ModifyTerrain/.templates/computation {\myUnderscore}extents.xlsx} are not readable.
		\item[\textit{Remedy}] Follow the instructions in Sec.~\ref{sec:mtsetreaches} for correct reach definitions.\\
	\end{itemize}
	
	\item[$\triangleright$]\textbf{\texttt{ERROR: Feature identification failed. Using default layout.}}
	\begin{itemize}
		\item[\textit{Cause}\hspace{0.27cm}] Raised by \pythoninline{choose_ref_layout(self, feature_type)} of \textit{MaxLifespan}'s \pythoninline{Mapper} class in \texttt{MaxLifespan /cMapActions.py} when there no layout could be assigned to the \pythoninline{feature_type} argument. The \pythoninline{feature_type} argument is not either \pythoninline{"terraforming"}, \pythoninline{"plantings"}, \pythoninline{"bioengineering"}, or \pythoninline{"maintenance"}.
		\item[\textit{Remedy}] -- If code was modified: Ensure that the new feature set can be recognized by the \pythoninline{choose_ref_layout(self, feature_type)} function. If needed, expand the \pythoninline{if} statement by the new feature set.\\
													 -- Check consistency of suspected lifespan/design rasters, the correctness of lifespan/design input directory definitions (Sec.~\ref{sec:actquick}) and if needed re-run lifespan/design Raster Maker.\\
	\end{itemize}
	
	\item[$\triangleright$]\textbf{\texttt{ERROR: FEAT SHORTNAME contains non-valid data or is empty.}}
	\begin{itemize}
		\item[\textit{Cause}\hspace{0.27cm}] Raised by \pythoninline{get_design_data(self)} in \texttt{MaxLifespan/cActionAssessment.py} when the feature \pythoninline{shortname} raster is empty or the \pythoninline{shortname} itself does not match the code conventions.
		\item[\textit{Remedy}] -- If code was modified: Review code modifications and ensure to define feature \pythoninline{shortname}s as listed in Sec.~\ref{sec:featoverview}. If a new feature was added, it also needs to be appended in the container lists (\pythoninline{self.id_list, self.threshold_cols, self.name_list}) of the \pythoninline{Feature()} class in \texttt{.site{\myUnderscore}packages/riverpy/cDefinit ions.py}. A new feature also requires modifications of the \texttt{RiverArchitect/LifespanDesign/.templates/threshold{\myUnderscore}values.xlsx} spreadsheet (Sec.~\ref{sec:modthresh}), in line with the column state in the \pythoninline{self.threshold_cols} list of the \pythoninline{Feature()} class.\\
			 -- Check consistency of suspected lifespan/design rasters, the correctness of lifespan/design input directory definitions (Sec.~\ref{sec:actquick}) and if needed re-run lifespan/design Raster Maker.\\
	\end{itemize}
	
	\item[$\triangleright$] \textbf{\texttt{ERROR: Function analysis{\myUnderscore}call received bad arguments.}}
	\begin{itemize}
		\item[\textit{Cause}\hspace{0.27cm}] The \pythoninline{analysis_call(*args)} method in \texttt{LifespanDesign/feature{\myUnderscore}analysis.py} causes this error when it is not able to assign an analysis function based on the provided \pythoninline{parameter_name}. It may come along with \texttt{ERROR: .cache folder in use.} or after changes have been effected in the code.
		\item[\textit{Remedy}] If the \texttt{.cache} folder is in use, delete it manually (works sometimes only after logging of and on). If the error occurs after code modifications, make sure that the \pythoninline{self.parameter_list}s of features (Sec.~\ref{sec:add-feat}) has valid entries that occur in \pythoninline{analysis_call(*args)} (\texttt{LifespanDesign/feature{\myUnderscore}analysis.py}) and that valid function names exist in \texttt{LifespanDesign/cLifespanDesignAnalysis.py} (Sec.~\ref{sec:add-ana}).\\
	\end{itemize}
	
	\item[$\triangleright$]\textbf{\texttt{ERROR: Incoherent data in RAS (raster comparison).}}
	\begin{itemize}
		\item[\textit{Cause}\hspace{0.27cm}] Raised by \pythoninline{compare_raster_set(self, ...)} function of the \pythoninline{ArcPyAnalysis()} class in \texttt{LifespanDesign/cLifespanDesignAnalysis.py} when the provided input raster \texttt{RAS} (hydraulic or scour fill or morphological units) are invalid.
		\item[\textit{Remedy}] --  Manually open the concerned \texttt{RAS} raster and ensure that it complies with the requirements for input rasters stated in Sec.~\ref{sec:input}.\\
						 --  Verify that the Rasters defined in \texttt{LifespanDesign/.templates/input{\myUnderscore}definitions.inp} (lines 8 to 18) correspond to the GRID raster names in the select conditions folder in \texttt{01{\myUnderscore}Conditions/}.\\
	\end{itemize}	
	
	\item[$\triangleright$]\textbf{\texttt{ERROR: Input file not available.}}
	\begin{itemize}
		\item[\textit{Cause}\hspace{0.27cm}] Raised by \pythoninline{get_line_entries(self, line_no)} function of the \pythoninline{Info()} class in \texttt{LifespanDesign/cRead InpLifespan.py} when it cannot access input files.
		\item[\textit{Remedy}] -- Ensure that the file \texttt{LifespanDesign/.templates/input{\myUnderscore}definitions.inp} exists in the directory \texttt{LifespanDesign/.templates/} corresponding to the definitions in Sec.~\ref{sec:inpfile}.\\
								-- Ensure that the file \texttt{mapping.inp} exists in the directory \texttt{LifespanDesign/.templates/} corresponding to the definitions in Sec.~\ref{sec:inpmaps}.\\
								-- In case of doubts: Replace \texttt{LifespanDesign/.templates/input{\myUnderscore}definitions.inp} and \texttt{mapping.inp} with the original files and re-apply modifications strictly following Sec.~\ref{sec:inp}.\\
	\end{itemize}
	
	\item[$\triangleright$]\textbf{\texttt{ERROR: Insufficient data. Check raster consistency and add more flows(?).}}
	\begin{itemize}
		\item[\textit{Cause}\hspace{0.27cm}] The \pythoninline{compare_raster_set(self, raster_set, threshold)} function in \texttt{LifespanDesign/cLifespan DesignAnalysis.py} raises this error when insufficient hydraulic rasters are provided or when the provided hydraulic rasters have inconsistent data.
		\item[\textit{Remedy}] -- Make sure to provide at least two pairs of hydraulic (\texttt{u} and \texttt{h}) rasters that correspond to two different discharges (one \texttt{u} and one \texttt{h} raster per discharge).\\
													 -- As a rule of thumb: the more hydraulic rasters provided, the better are the lifespan maps. However, for reasons of consistency, the maximum number of hydraulic rasters in six per \texttt{u} and one \texttt{h}, i.e., six lifespans.\\
													 -- Verify raster and corresponding lifespan definitions in \texttt{LifespanDesign/.templates/input{\myUnderscore}d efinitions.inp} (Sec. ~\ref{sec:inpfile}).\\
	\end{itemize}
	
	\item[$\triangleright$]\textbf{\texttt{ERROR: Invalid cell assignment for discharge / rasters.}}
	\begin{itemize}
		\item[\textit{Cause}\hspace{0.27cm}] Error raised by the \pythoninline{make_condition_xlsx(self, fish_sn)} function of the \pythoninline{Write()} class in \texttt{HabitatEval uation/cHabitatIO.py}) when it cannot write discharge values to \texttt{RiverArchitect/Habit atEvaluation/AUA/\textit{condition}{\myUnderscore}\textit{fill}.xlsx}.
		\item[\textit{Remedy}] Ensure that the flow duration curve is well defined (see Sec.~\ref{sec:hemakehsi}) and that \texttt{RiverArchitect/Habitat Evaluation/AUA/\textit{condition}{\myUnderscore}\textit{fill}.xlsx} is not used by any other application.\\
	\end{itemize}	
	
	\item[$\triangleright$]\textbf{\texttt{ERROR: Invalid feature names for column headers.}}
	\begin{itemize}
		\item[\textit{Cause}\hspace{0.27cm}] Error raised by the \pythoninline{write_volumes(self, ...)} function of the \pythoninline{Writer()} class in \texttt{.site{\myUnderscore}packages/riverpy/cTerrainIO.py}, when the \texttt{template} sheet in the output (template) workbook (\texttt{Modify Terrain/Output/Spreadsheets/\textit{condition}{\myUnderscore}volumes.xlsx} or \texttt{...volume{\myUnderscore}template.xlsx}) has inconsistent feature (short-) names.
		\item[\textit{Remedy}] Ensure that \texttt{ModifyTerrain/Output/Spreadsheets/\textit{condition}{\myUnderscore}volumes.xlsx} or \\\texttt{...volume{\myUnderscore}template.xlsx} contain consistent header names (Sec.~\ref{sec:mtoutspread}) corresponding to the definitions in Sec.~\ref{sec:featoverview}.\\
	\end{itemize}
	
	\item[$\triangleright$]\textbf{\texttt{ERROR: Invalid feature ID.}}
	\begin{itemize}
		\item[\textit{Cause}\hspace{0.27cm}] Error raised by the \pythoninline{__init__(self, ...)} function of the \pythoninline{ThresholdDirector()} class in \texttt{/LifespanDesign/cThresholdDirector.py}, when the feature IDs (shortnames) in \texttt{/LifespanDesign/.templates/threshold{\myUnderscore}values.xlsx} are incorrectly defined.
		\item[\textit{Remedy}] -- Ensure correct definitions in \texttt{/LifespanDesign/.templates/threshold{\myUnderscore}values.xlsx} (Sec.~\ref{sec:modthresh}).\\
								-- Consider replacing corrupted threshold workbooks with the original file.\\
	\end{itemize}
	
	\item[$\triangleright$]\textbf{\texttt{ERROR: Invalid file name or data.}}
	\begin{itemize}
		\item[\textit{Cause}\hspace{0.27cm}] Error raised by the \pythoninline{save_close_wb(self, *args)} function of the \pythoninline{Write()} class in \texttt{HabitatEvaluation/cHabitatIO.py}) or \texttt{ProjectMaker/cIO.py}) when it cannot save \texttt{RiverArchitect/Habi tatEvaluation/AUA/\textit{condition}{\myUnderscore}\textit{fill}.xlsx} or a copy of the cost master workbook.
		\item[\textit{Remedy}] -- \textit{HabitatEvaluation}: Close all applications that may use \texttt{\textit{condition}{\myUnderscore}\textit{fill}.xlsx} and ensure that its template exists. Detailed information on \textit{HabitatEvaluation} workbook outputs are available in Sec.~\ref{sec:hemakehsi}.\\
		-- \textit{ProjectMaker}:  Close all applications that may use the cost master workbook (\texttt{\emph{REACH}{\myUnderscore}\emph{stn}{\myUnderscore}\emph{costs}{\myUnderscore}\emph{ver}\\\emph{sion}.xlsx}) and ensure that it exists. Detailed information are available in Sec.~\ref{sec:pmcq}.\\
	\end{itemize}
	
	\item[$\triangleright$]\textbf{\texttt{ERROR: Invalid interpolation data type (type(Q flowdur) ==  ...)}}
	\begin{itemize}
		\item[\textit{Cause}\hspace{0.27cm}] Raised by \pythoninline{interpolate_flow_exceedance(self, Q_value)} of \textit{HabitatEvaluation}'s \pythoninline{FlowAssessment()} class in \texttt{HabitatEvaluation/cHSI.py} when the flow duration curve contains invalid data.
		\item[\textit{Remedy}] Ensure the correct setup of the used flow duration curve in \texttt{HabitatEvaluation/FlowDurationCu rves/}. The file structure must correspond to that of the provided template \texttt{flow{\myUnderscore}duration{\myUnderscore}template.xlsx}. Review Sec.~\ref{sec:hemakehsi} for details.\\
	\end{itemize}
	
	\item[$\triangleright$]\textbf{\texttt{ERROR: Invalid x-y coordinates in mapping.inp}}
	\begin{itemize}
		\item[\textit{Cause}\hspace{0.27cm}] The \pythoninline{make_pdf_maps(self, *args)} function of the \pythoninline{Mapper} class in \texttt{MaxLifespan/cMapActions.py} raises this error when \texttt{MaxLifespan/.templates/mapping.inp} contains invalid map definitions (extents).
		\item[\textit{Remedy}] Ensure the definitions of \texttt{MaxLifespan/.templates/mapping.inp} analog to the descriptions of the \textit{LifespanDesign} module in Sec.~\ref{sec:inpmaps}.\\
	\end{itemize}
	
	\item[$\triangleright$]\textbf{\texttt{ERROR: Invalid x-y coordinates in reach spreadsheet.}}
	\begin{itemize}
		\item[\textit{Cause}\hspace{0.27cm}] The \pythoninline{map_custom(self, input_ras_dir, *args)}, \pythoninline{map_reach(self, reach_id, feature_id, *args)} functions of the \pythoninline{Mapper} class in \texttt{ModifyTerrain/cMapModifiedTerrain.py} raises this error when the reach definition spreadsheet (\texttt{ModifyTerrain/.templates/computation{\myUnderscore}extents.xlsx}) contains invalid coordinates.
		\item[\textit{Remedy}] Ensure the definitions in \texttt{ModifyTerrain/.templates/computation{\myUnderscore}extents.xlsx} correspond to the descriptions in Sec.~\ref{sec:mtsetreaches}, using consistent coordinate and unit systems.\\
	\end{itemize}
	
	\item[$\triangleright$]\textbf{\texttt{ERROR: Invalid xy-extents.}}
	\begin{itemize}
		\item[\textit{Cause}\hspace{0.27cm}] The \pythoninline{map_custom(self, input_ras_dir, *args)}, \pythoninline{map_reach(self, reach_id, feature_id, *args)} fun-\\ctions of \pythoninline{Mapper()} class (\texttt{ModifyTerrain/cMapModifiedTerrain.py} raises this error message when the customary defined DEM raster is corrupted.
		\item[\textit{Remedy}] Ensure that customary defined DEM rasters are non-empty rasters with coherent coordinate and units systems and that rasters are in the stated directory for customary DEMs (default directory: \texttt{ModifyTerrain/Input/DEM/\textit{condition}/}), as described in Sec.~\ref{sec:mtcustdem}.\\
	\end{itemize}
	
	\item[$\triangleright$]\textbf{\texttt{ERROR: Invalid keyword for feature type.}}
	\begin{itemize}
		\item[\textit{Cause}\hspace{0.27cm}] The \pythoninline{Manager} class in \texttt{MaxLifespan/cFeatureActions.py} raises this error when it received a \pythoninline{feature_type} argument that is not \pythoninline{"terraforming"}, \pythoninline{"plantings"}, \pythoninline{"bioengineering"}, or \pythoninline{"maintenance"}. The error may occur either after code modifications or when \pythoninline{geo_file_maker(condition, feature_type, *args)} in \texttt{MaxLif espan/action{\myUnderscore}planner.py} was executed as standalone or imported as a package in an external application.
		\item[\textit{Remedy}] -- Ensure that code extensions comply with coding conventions and instructions in Sec.~\ref{sec:actcode}. \\
							   -- Ensure that external calls of \pythoninline{geo_file_maker(condition, feature_type, *args)} contain an acceptable \pythoninline{feature_type}, i.e., \pythoninline{feature_type=} either \pythoninline{"terraforming"}, \pythoninline{"plantings"}, \pythoninline{"bioengineering"}, or \pythoninline{"maintenance"}.\\
	\end{itemize}
	
	\item[$\triangleright$]\textbf{\texttt{ERROR: Lifespan data fetch failed.}}
	\begin{itemize}
		\item[\textit{Cause}\hspace{0.27cm}] The \pythoninline{get_lifespan_data(self)} or \pythoninline{get_design_data(self)} function of the \pythoninline{ArcPyContainer} class in \texttt{MaxLif espan/cActionAssessment.py} raise this error when it could not retrieve lifespan or design maps from the defined lifespan/design input directory.
		\item[\textit{Remedy}] -- Check lifespan/design folder definitions (review Sec.~\ref{sec:actquick}).\\
							   -- Ensure that lifespan and/or design rasters are in the defined folder.\\
	\end{itemize}
	
	\item[$\triangleright$]\textbf{\texttt{ERROR: Mapping failed.}}
	\begin{itemize}
		\item[\textit{Cause}\hspace{0.27cm}] The function \pythoninline{make_pdf_maps(self, *args)} (\texttt{LifespanDesign/cMapLifespanDesign.py} or \texttt{Act ionPlanner/cMapActions.py}) or \pythoninline{map_custom(self, input_ras_dir, *args)}, \pythoninline{map_reach(self,}\\
	\pythoninline{reach_id, feature_id, *args)} (\texttt{ModifyTerrain/cMapModifiedTerrain.py}) raise this error message when it could not create \texttt{PDF} maps. 
		\item[\textit{Remedy}] -- \textit{LifespanDesign (1)}: The layout files in \texttt{LifespanDesign/Output/Mapping/\textit{condition}/Layou ts/} are either corrupted or non-existent. Re-run \pythoninline{Layout Maker} or successively re-run \pythoninline{Raster Maker} and \pythoninline{Layout Maker}. Follow exactly the instructions for preparing map files (see Sec.~\ref{sec:outmaps}).\\
		 -- \textit{LifespanDesign (2)}: Make sure that the file \texttt{legend.ServerStyle} exists in \texttt{LifespanDesign/Output/Mapping/.ReferenceLayouts}\\
		 -- \textit{MaxLifespan}: Ensure consistent layout files in \texttt{MaxLifespan/.templates/layouts/} (see Sec.~\ref{sec:actoutmaps}) and trace back earlier warning and error messages.\\
		 -- \textit{ModifyTerrain}: Ensure consistent layout files in \texttt{ModifyTerrain/Input/Layouts/} (see Sec.~\ref{sec:mtlyt}) and trace back earlier warning and error messages.\\
	\end{itemize}
	
	\item[$\triangleright$]\textbf{\texttt{ERROR: Map layout preparation failed.}}
	\begin{itemize}
		\item[\textit{Cause}\hspace{0.27cm}] The \pythoninline{prepare_layout(self)} functions of \pythoninline{Mapper()} classes (\texttt{LifespanDesign/cMapLifespanDesi gn.py}, \texttt{MaxLifespan/cMapActions.py} or \texttt{ModifyTerrain/cMapModifiedTerrain.py}) raise this error message when they encounter problems with either the provided rasters or layout (\texttt{.mxd}) files.
		\item[\textit{Remedy}] -- \textit{LifespanDesign}: If a layout (\texttt{.mxd}) in \texttt{LifespanDesign/Output/Mapping/.ReferenceLay outs/} was modified, ensure similar layer structures in the \texttt{.mxd} files corresponding to the existing templates  (default directory: \texttt{LifespanDesign/Output/Rasters/\textit{condition}/}) or layout templates (\texttt{.mxd} files in \texttt{LifespanDesign/Output/Mapping/.ReferenceLayouts}).\\
							  -- \textit{MaxLifespan}: Ensure that all relevant \texttt{.mxd} layouts (\pythoninline{"terraforming"}, \pythoninline{"plantings"}, \pythoninline{"bioengineering"}, or \pythoninline{"maintenance"}) are contained in the \texttt{MaxLifespan/.templates/layouts/} directory (see also Sec.~\ref{sec:actoutmaps}). If needed, add new layouts after code modifications (Sec.~\ref{sec:actcode}).\\
								-- \textit{ModifyTerrain}: Ensure that a layout template exists (explanations in Sec.~\ref{sec:mtlyt}).\\
	\end{itemize}
	
	\item[$\triangleright$]\textbf{\texttt{ERROR: Mapping could not assign xy-values. Undefined zoom.}}
	\begin{itemize}
		\item[\textit{Cause}\hspace{0.27cm}] Error raised by the \pythoninline{zoom2map(self, xy)} functions of the \pythoninline{Mapper()} classes (\texttt{LifespanDesign/cMapLi fespanDesign.py}, \texttt{MaxLifespan/cMapActions.py} or \texttt{ModifyTerrain/cMapModified\\Terrain.py}) when it receives a bad format of $x$-$y$ values.
		\item[\textit{Remedy}] -- Ensure the correct format of \texttt{mapping.inp} (LifespanDesign or \textit{MaxLifespan} module) corresponding to the definitions in Sec.~\ref{sec:inpmaps}.\\
													 -- Ensure correct setup of \texttt{ModifyTerrain/.templates/computation{\myUnderscore}extents.xlsx}\\(Sec.~\ref{sec:mtsetreaches}).\\
	\end{itemize}
	
	\item[$\triangleright$]\textbf{\texttt{ERROR: Missing (or wrong format of) raster input definitions.}}
	\begin{itemize}
		\item[\textit{Cause}\hspace{0.27cm}] Raised by \pythoninline{get_line_entries(self, line_no)} function of the \pythoninline{Info()} class in \texttt{LifespanDesign/cRead InpLifespan.py} when \\ \texttt{LifespanDesign/.templates/input{\myUnderscore}definitions.inp} is corrupted.
		\item[\textit{Remedy}] Ensure that the file \texttt{LifespanDesign/.templates/input{\myUnderscore}definitions.inp} exists in the directory \texttt{LifespanDesign/.templates/} corresponding to the definitions in Sec.~\ref{sec:inpfile}. In case of doubts: Replace \texttt{LifespanDesign/.templates/input{\myUnderscore}definitions.inp} with the original file and re-apply modifications strictly following Sec.~\ref{sec:inp}.\\
	\end{itemize}
	
	\item[$\triangleright$]\textbf{\texttt{ERROR: Multiple openings of Fish.xlsx. Close all office apps ...}}
	\begin{itemize}
		\item[\textit{Cause}\hspace{0.27cm}] Raised by the \pythoninline{assign_fish_names(self)} function of the \pythoninline{Fish()} class in \texttt{HabitatEvaluation/cFish.py} when \\ \texttt{HabitatEvaluation/.templates/Fish.xlsx} is opened by another program or non-existent.
		\item[\textit{Remedy}] Ensure that the file \texttt{HabitatEvaluation/.templates/Fish.xlsx} exists and close any software that may use the workbook.\\
	\end{itemize}
	
	\item[$\triangleright$]\textbf{\texttt{ERROR: No HSI assigned for parameter type ...}}
	\begin{itemize}
		\item[\textit{Cause}\hspace{0.27cm}] Raised by the \pythoninline{get_hsi_curve(self, species, lifestage, par)} function of the \pythoninline{Fish()} class in \texttt{Habitat Evaluation/cFish.py} when \\ \texttt{HabitatEvaluation/.templates/Fish.xlsx} it expected a habitat suitability curve for \texttt{par}, but it could not find values..
		\item[\textit{Remedy}] Ensure that the file \texttt{HabitatEvaluation/.templates/Fish.xlsx} has valid contents according to Sec.~\ref{sec:hefish}.\\
	\end{itemize}
	
	\item[$\triangleright$]\textbf{\texttt{ERROR: No custom (DEM/feature) raster found.}}
	\begin{itemize}
		\item[\textit{Cause}\hspace{0.27cm}] The \pythoninline{map_custom(self, input_ras_dir, *args)} function of \pythoninline{Mapper()} class (\texttt{ModifyTerrain/cMapMo difiedTerrain.py} raises this error message when it cannot find customary defined DEM rasters (default directory: \texttt{ModifyTerrain/Input/DEM/\textit{condition}/}).
		\item[\textit{Remedy}] Ensure that customary defined DEM rasters are in the stated directory for customary DEMs (default directory: \texttt{ModifyTerrain/Input/DEM/\textit{condition}/}), as described in Sec.~\ref{sec:mtcustdem}.\\
	\end{itemize}
		
	\item[$\triangleright$]\textbf{\texttt{ERROR: No HSI assigned for parameter type ...}}
	\begin{itemize}
		\item[\textit{Cause}\hspace{0.27cm}] Raised by the \pythoninline{get_hsi_curve(self, species, lifestage, par)} function of the \pythoninline{Fish()} class in \texttt{Habitat Evaluation/cFish.py} when \\ \texttt{HabitatEvaluation/.templates/Fish.xlsx} it expected a habitat suitability curve for \texttt{par}, but it could not find values..
		\item[\textit{Remedy}] Ensure that the file \texttt{HabitatEvaluation/.templates/Fish.xlsx} has valid contents according to Sec.~\ref{sec:hefish}.\\
	\end{itemize}	
	
	\item[$\triangleright$]\textbf{\texttt{ERROR: No layout template found (feature ID: FEAT.}}
	\begin{itemize}
		\item[\textit{Cause}\hspace{0.27cm}] Error raised by the \pythoninline{choose_ref_layout(self, feature_id, volume_type)} function of the \pythoninline{Mapper()} class in \texttt{ModifyTerrain/cMapModifiedTerrain.py} when it cannot match layout files in \texttt{ModifyTer rain/Input/Layouts/\textit{condition}/} for the feature shortname \pythoninline{FEAT} and a \pythoninline{neg} or \pythoninline{pos} string.
		\item[\textit{Remedy}] Ensure that a layout is available in \texttt{ModifyTerrain/Input/Layouts/\textit{condition}/} according to the descriptions in Sec.~\ref{sec:mtlyt}.\\
	\end{itemize}
	
	\item[$\triangleright$]\textbf{\texttt{ERROR: \textit{PAR} - raster copy to Output/Rasters folder failed.}}
	\begin{itemize}
		\item[\textit{Cause}\hspace{0.27cm}] The \texttt{.cache} folder does not exist or does not contain GRID rasters or the output folder is not accessible. This error is likely to occur when other errors occurred previously. 
		\item[\textit{Remedy}] -- Follow trouble shooting of other error messages and re-run.\\
							   -- Avoid modifications of any folder in the code directory while the program is running, in particular,\\ \texttt{.cache}, \texttt{01{\myUnderscore}Conditions/}, \texttt{LifespanDesign/Output/Rasters/} and \texttt{Lifespan Design/Output/Mapping/}.\\
	\end{itemize}
	
	\item[$\triangleright$]\textbf{\texttt{ERROR: Raster copy to Output folder failed.}}
	\begin{itemize}
		\item[\textit{Cause}\hspace{0.27cm}] The \pythoninline{save_rasters(self)} function of the \pythoninline{ModifyTerrain()} class in \texttt{ModifyTerrain/cModifyTerra in.py} raises this error when saving a terrain differences or new DEM raster failed.
		\item[\textit{Remedy}] Refer to the error \texttt{ERROR: Raster could not be saved.} message.\\
	\end{itemize}
	
	\item[$\triangleright$]\textbf{\texttt{ERROR: Raster could not be saved.}}
	\begin{itemize}
		\item[\textit{Cause}\hspace{0.27cm}] The \pythoninline{save_rasters(self)} function of the \pythoninline{ModifyTerrain()} class in \texttt{ModifyTerrain/cModifyTerra in.py} raises this error when a terrain differences or new DEM raster is corrupted.
		\item[\textit{Remedy}] Potential reasons for corrupted rasters are:\\
													 -- The computed volume difference or new DEM raster is empty or contains \texttt{NoData} pixels only. The design parameters or raster of the concerned feature need to be reviewed.\\
													 -- The \texttt{ModifyTerrain/.cache/} folder is locked by another program. Close potential applications, and if necessary, reboot the system.
													 -- If \texttt{ModifyTerrain/.cache/} was not empty before the module execution, error may occur. Manually delete \texttt{ModifyTerrain/.cache/} if it still exists after a run task.\\
													 -- The directory \texttt{ModifyTerrain/Output/Rasters/\textit{condition}/} was deleted or it is locked by another program. Ensure that the directory exists and no other program uses \texttt{ModifyTerrain/Output/Rasters/\textit{condition}/} or its contents.\\
	\end{itemize}
	\item[]
	
	\item[$\triangleright$]\textbf{\texttt{ERROR: Raster identification failed. Omitting layout creation of ...}}
	\begin{itemize}
		\item[\textit{Cause}\hspace{0.27cm}] Error message raised by the \pythoninline{choose_ref_layout(self, raster_name)} function in \texttt{LifespanDesign/cMapLifespanDesign.py} when it cannot assign a layout template from \texttt{LifespanDesign/Out put/Mapping/.ReferenceLayouts} to a raster (default storage directory: \texttt{LifespanDesign/Out put/Rasters/\textit{condition}/}).
		\item[\textit{Remedy}] -- If a layout (\texttt{.mxd}) in \texttt{LifespanDesign/Output/Mapping/.ReferenceLayouts/} was modified, make sure to implement changes also in the \pythoninline{choose_ref_layout(self, raster_name)} function (\texttt{Life spanDesign/cMapLifespanDesign.py}).\\
													 -- If a new output raster type results from modifications or extensions of the parameters, analysis or feature methods (Sections~\ref{sec:add-par}, ~\ref{sec:add-ana} and ~\ref{sec:add-feat}, respectively), ensure that the conditional phrases in \pythoninline{choose_ref_layout(self, raster_name)} (\texttt{LifespanDesign/cMapLifespanDesign.py}) can identify it and assign an existing layout (\texttt{.mxd}) from \texttt{LifespanDesign/Output/Mapping/.Referen ceLayouts/} .\\
	\end{itemize}
	
	\item[$\triangleright$]\textbf{\texttt{ERROR: Received request for volume calculation but not input directory ...}}
	\begin{itemize}
		\item[\textit{Cause}\hspace{0.27cm}] The \pythoninline{__call__(self, *args)} function of the \pythoninline{ModifyTerrain()} class in \texttt{ModifyTerrain/cModifyTer rain.py} raises this error when it received \pythoninline{args[0] = True} (enable volume calculator only), but no input directory for a modified terrain is given (missing \pythoninline{args[1] = DIRECTORY}). This error may occur if the code was modified or called externally.
		\item[\textit{Remedy}] Ensure that the input directory of manually modified (Customary Feature) raster DEMs (default: \texttt{Modify Terrain/Input/DEM/\textit{condition}/}) is correctly passed to the \textit{ModifyTerrain} object.\\
	\end{itemize}
	
	\item[$\triangleright$]\textbf{\texttt{ERROR: Scale is not INT. Substituting scale: 2000.}}
	\begin{itemize}
		\item[\textit{Cause}\hspace{0.27cm}] Raised by \pythoninline{get_map_scale(self)} function of the \pythoninline{Info()} class in either \texttt{LifespanDesign/cReadInpLi fespan.py} or \texttt{MaxLifespan/cReadActionInput.py} when it cannot interpret the value assigned to the map scale.
		\item[\textit{Remedy}] Ensure that the file \texttt{mapping.inp} (in \texttt{LifespanDesign/.templates/} or \texttt{MaxLifespan /.templates/}) has a correct assignment of the map scale according to the descriptions in Sec.~\ref{sec:inpmaps}.\\
	\end{itemize}
	
	\item[$\triangleright$]\textbf{\texttt{ERROR: Shapefile conversion failed.}}
	\begin{itemize}
		\item[\textit{Cause}\hspace{0.27cm}] Raised by \pythoninline{calculate_wua(self)} of \textit{HabitatEvaluation}'s \pythoninline{CHSI()} class in \texttt{HabitatEvaluation/cHSI.py} when it could not convert the CHSI raster to a shapefile.
		\item[\textit{Remedy}] -- Ensure that the AUA threshold has a meaningful value between 0.0 and 1.0 (Sec.~\ref{sec:heintro}).\\
							 -- Ensure that neither the directory \texttt{HabitatEvaluation/.cache/} nor the directory \texttt{HabitatEval uation/AUA/} or their contents are in use by other programs.\\
							 -- Review the input settings according to Sec.~\ref{sec:hequick}.\\
							 -- Follow up earlier error messages.\\
	\end{itemize}
	
	\item[$\triangleright$]\textbf{\texttt{ERROR: TEMPLATE sheet does not exist.}}
	\begin{itemize}
		\item[\textit{Cause}\hspace{0.27cm}] Error raised by the \pythoninline{write_volumes(self, ...)} function of the \pythoninline{Writer()} class in \texttt{.site{\myUnderscore}packages/riverpy/cTerrainIO.py}) or the \pythoninline{make_condition_xlsx(self, fish_sn)} of the \pythoninline{Write()} class in \texttt{Habi tatEvaluation/cHabitatIO.py}) when the \texttt{template} sheet in the output (template) workbooks (\texttt{Modi fyTerrain/Output/Spreadsheets/\textit{condition}{\myUnderscore}volumes.xlsx}, \texttt{...volume{\myUnderscore}tem plate.xlsx} or \texttt{HabitatEvaluation/.templates/Q{\myUnderscore}def{\myUnderscore}hab{\myUnderscore}template{\myUnderscore}....xlsx}) are corrupted.
		\item[\textit{Remedy}] -- \textit{ModifyTerrain}: Ensure that \texttt{ModifyTerrain/Output/Spreadsheets/\textit{condition}{\myUnderscore}volumes.xlsx} or\\ \texttt{...volume{\myUnderscore}template.xlsx} contain the \texttt{template} sheet (Sec.~\ref{sec:mtoutspread}).\\
		-- \textit{HabitatEvaluation}: Ensure that \texttt{HabitatEvaluation/.templates/Q{\myUnderscore}def{\myUnderscore}hab{\myUnderscore}template{\myUnderscore}....xlsx} contains the \texttt{summary} sheet; re-install the templates if necessary.\\
	\end{itemize}
	
	\item[$\triangleright$] \textbf{\texttt{ERROR: u/h/hyd--raster analysis does not accept ras{\myUnderscore}\textit{name} raster.}}
	\begin{itemize}
		\item[\textit{Cause}\hspace{0.27cm}] Internal programming error: A parameter module called a raster which does not match the batch processing hierarchy.
		\item[\textit{Remedy}] Move new model downward in the processing hierarchy and avoid calling an u/h/hyd--raster with the optional argument \texttt{raster{\myUnderscore}info}.\\
	\end{itemize}
	
	\item[$\triangleright$]\textbf{\texttt{ERROR: Volume value assignment failed.}}
	\begin{itemize}
		\item[\textit{Cause}\hspace{0.27cm}] Error raised by the \pythoninline{write_volumes(self, ...)} function of the \pythoninline{Writer()} class in \texttt{.site{\myUnderscore}packages/riverpy/cTerrainIO.py}) when it received invalid volume data.
		\item[\textit{Remedy}] Ensure that no other program uses \texttt{ModifyTerrain/Output/Spreadsheets/\textit{condition}{\myUnderscore}vol ume.xlsx} and trace back earlier errors (modified DEM rasters may be corrupted).\\
	\end{itemize}
	
	\item[$\triangleright$]\textbf{\texttt{ERROR: Writing failed.}}
	\begin{itemize}
		\item[\textit{Cause}\hspace{0.27cm}] Error raised by the \pythoninline{write_volumes(self, ...)} function of the \pythoninline{Writer()} class in \texttt{.site{\myUnderscore}packages/riverpy/cTerrainIO.py}) when the \texttt{template} it could not add new sheets in \texttt{ModifyTerrain/Output/Spreadsheets/\textit{condition}{\myUnderscore}volumes.xlsx} or write to copies of \texttt{ModifyTerrain/Output/Spreadsheets/volume{\myUnderscore}template.xlsx}.
		\item[\textit{Remedy}] See error message \texttt{ERROR: Failed to create WORKBOOK}.\\
	\end{itemize}
	
	\item[$\triangleright$]\textbf{\texttt{ERROR: Wrong format of lifespan list (.inp)}}
	\begin{itemize}
		\item[\textit{Cause}\hspace{0.27cm}] Raised by \pythoninline{lifespan_read(self)} (in \texttt{LifespanDesign/cReadInpLifespan.py}) when the lifespan list in\\ \texttt{LifespanDesign/.templates/input{\myUnderscore}definitions.inp} has a wrong format or is empty.
		\item[\textit{Remedy}] Ensure that the file \texttt{LifespanDesign/.templates/input{\myUnderscore}definitions.inp} (in \texttt{Lifespan Design/.templates/}) contains a lifespan list (return periods list) with not more than six comma-separated entries according to the definitions in Sec.~\ref{sec:inpfile}.\\
	\end{itemize}
	
	\item[$\triangleright$]\textbf{\texttt{ExceptionERROR: (arcpy) [...].}}
	\begin{itemize}
		\item[\textit{Cause}\hspace{0.27cm}] The error is raised if any \pythoninline{arcpy} application of any module encountered problems; e.g., the \pythoninline{analysis_...} and \pythoninline{design_...} functions in \texttt{LifespanDesign/cLifespanDesignAnalysis.py} raise this error when raster calculations could not be performed. Missing rasters, bad raster assignments or bad raster calculation expressions are possible reasons. The error can also occur when the \texttt{Spatial} license is not available.
		\item[\textit{Remedy}] -- Make sure that a \texttt{Spatial} license is available.\\
													 -- Trace back previous error and warning messages.\\
													 -- Verify raster calculation expressions in concerned \pythoninline{analysis_...} and \pythoninline{design_...} functions\\ \hspace{0.3cm} (\texttt{LifespanDesign/cLifespanDesignAnalysis.py}).\\
													 -- Verify raster definitions in concerned \pythoninline{analysis_...} and \pythoninline{design_...} functions (\texttt{LifespanDesign/cLifespanDesignAnalysis.py}).\\
													 -- Verify raster definitions of used parameters (\texttt{cParameters.py} and input files \texttt{*.inp} according to Sec.~\ref{sec:inp}).\\
													 -- If further system errors are stated, trace back error messages.\\
	\end{itemize}
	
	\item[$\triangleright$]\textbf{\texttt{ExceptionERROR: Cannot find package files [...].}}
	\begin{itemize}
		\item[\textit{Cause}\hspace{0.27cm}] The program cannot retrieve the listed internal files.
		\item[\textit{Remedy}] Check the installation of the package and its file structure according to Sec.~\ref{sec:req}.\\
	\end{itemize}
	
	\item[$\triangleright$]\textbf{\texttt{ExceptionERROR: Cannot open reference (condition) ...}}
	\begin{itemize}
		\item[\textit{Cause}\hspace{0.27cm}] Raised by the \pythoninline{ModifyTerrain()} class (\pythoninline{__init__(self,condition, feature_type, *args)}) in \texttt{ModifyTer rain/cModifyTerrain.py} when it cannot find a \texttt{...} raster in \texttt{01{\myUnderscore}Conditions/\textit{condition}/} (or other user defined input directory), where \texttt{...} is either a \texttt{dem} or a \texttt{wt{\myUnderscore}depth{\myUnderscore}base} raster. A \texttt{wt{\myUnderscore}depth{\myUnderscore}base} raster is required for automated terrain modification after grading and/or widen features. 
		\item[\textit{Remedy}] Ensure that the missing raster (\texttt{dem} or a \texttt{wt{\myUnderscore}depth{\myUnderscore}base}) exists in \texttt{01{\myUnderscore}Conditions/\textit{condition}/}, or if applies, the user defined input directory. If no \texttt{wt{\myUnderscore}depth{\myUnderscore}base} raster is available, the terrain modification of grading and/or widen features cannot be automated. In this case, consider adding a new DEM automation function (explained in Sec.~\ref{sec:addmtmod}) or modifying the DEM manually.\\
	\end{itemize}
	
	\item[$\triangleright$]\textbf{\texttt{ExceptionERROR: Could not find base raster for assigning lifespans.}}
	\begin{itemize}
		\item[\textit{Cause}\hspace{0.27cm}] Raised by \textit{MaxLifespan}'s \pythoninline{ArcPyContainer()} class (\pythoninline{__init__(self,condition, feature_type, *args)}) in \texttt{MaxLifespan/cActionAssessment.py} when it cannot find its zero raster template in \texttt{MaxLif espan/.templates/rasters/zeros}.
		\item[\textit{Remedy}] Follow the instructions for the error message \texttt{ExceptionERROR: Unable to create ZERO Ras ter. Manual intervention required:...} to manually create the \texttt{MaxLifespan/.templates/rasters/zeros} raster.\\
	\end{itemize}
	
	\item[$\triangleright$]\textbf{\texttt{ExceptionERROR: Could not retrieve zero raster from \textit{MaxLifespan}.}}
	\begin{itemize}
		\item[\textit{Cause}\hspace{0.27cm}] Raised by the \pythoninline{ModifyTerrain()} class (\pythoninline{__init__(self,condition, feature_type, *args)}) in \texttt{ModifyTer rain/cModifyTerrain.py} when it cannot find the zero raster template in \texttt{MaxLifespan/.templates/rasters/zeros}.
		\item[\textit{Remedy}] Follow the instructions for the error message \texttt{ExceptionERROR: Unable to create ZERO Ras ter. Manual intervention required:...} to manually create the \texttt{MaxLifespan/.templates/rasters/zeros} raster.\\
	\end{itemize}
	
	\item[$\triangleright$]\textbf{\texttt{ExceptionERROR: Missing fundamental packages (required: ...).}}
	\begin{itemize}
		\item[\textit{Cause}\hspace{0.27cm}] The listed (required) packages are not available.
		\item[\textit{Remedy}] Check installation of required packages and code structure files according to Sec.~\ref{sec:req}.\\
	\end{itemize}
	
	\item[$\triangleright$]\textbf{\texttt{ExceptionERROR: Unable to create ZERO Raster. Manual intervention required}}
	\begin{itemize}
		\item[\textit{Cause}\hspace{0.27cm}] \textit{MaxLifespan} failed to create a zero raster covering the computation area.
		\item[\textit{Remedy}] The raster creation needs to be manually made in \textit{ArcMap}'s Python interpreter (the external interpreter could not do the job and only the cuckoo from
California knows why). Thus, manually create the zeros raster as follows:
		\begin{enumerate}
			\item Launch \textit{ArcMap} and its implemented Python window (\texttt{Geoprocessing} dropdown menu: \texttt{Python}).
			\item Enter the following sequences (replace \pythoninline{REPLACE_...} according to the local environment):
\begin{python}
import os
from arcpy.sa import *
zero_ras_str = os.getcwd() + "\\.templates\\rasters\\zeros"
condition = "REPLACE_CONDITION"
base_dem = arcpy.Raster("REPLACE_PATH\\RiverArchitect\\LifespanDesign\\Input\\" + condition + "\\dem")
arcpy.gp.overwriteOutput = True
arcpy.env.extent = base_dem.extent
arcpy.env.workspace = "D:\\Python\\RiverArchitect\\LifespanDesign\\Input\\" + condition + "\\"
zero_ras = Con(IsNull(base_dem), 0, 0)
zero_ras.save(zero_ras_str)
\end{python}
			\item Close \textit{ArcMap}
		\end{enumerate}
	\end{itemize}
	
	\item[$\triangleright$]\textbf{\texttt{ExecuteERROR: (arcpy) [...].}}
	\begin{itemize}
		\item[\textit{Cause}\hspace{0.27cm}] Similar to \pythoninline{ExceptionERROR: (arcpy) ...}. The error is raised by \pythoninline{arcpy} applications of all modules; e.g., by the \pythoninline{analysis_...} and \pythoninline{design_...} functions in \texttt{LifespanDesign/cLifespanDesignAnalysis.py} or when raster calculations could not be performed. Missing rasters, bad raster assignments or bad raster calculation expressions are possible reasons. The error can also occur when the \texttt{Spatial} license is not available.
		\item[\textit{Remedy}] See \texttt{ExceptionERROR: (arcpy) [...]}\\
	\end{itemize}
	
	\item[$\triangleright$] \textbf{\texttt{WindowsError: [Error 32] The process cannot access the file because ...}}
	\begin{itemize}
		\item[\textit{Cause}\hspace{0.27cm}] Files in the \texttt{.cache}--folder or the \texttt{Output}--folder are used by another program.
		\item[\textit{Remedy}] -- Make sure that \texttt{ArcGIS} Desktop is not running.\\
													 -- Make sure that no other code copy (Python) uses these folders.\\
	\end{itemize}
\end{itemize}

\subsection{Warning messages}

\begin{itemize}
	\item[$\triangleright$]\textbf{\texttt{WARNING: .cache folder will be removed by package controls.}}
	\begin{itemize}
		\item[\textit{Cause}\hspace{0.27cm}] Raised by \pythoninline{clear_cache(self)} of \textit{HabitatEvaluation}'s \pythoninline{CHSI()} class in \texttt{HabitatEvaluation/cHSI.py} when it could not clear and remove the \texttt{.cache/} folder.
		\item[\textit{Remedy}] Ensure that no other software uses the temporary rasters stored in \texttt{HabitatEvaluation/.cache/}, and if necessary, delete the folder manually after quitting the module.\\
	\end{itemize}

	\item[]
	\item[$\triangleright$]\textbf{\texttt{WARNING: Bad value ( ... ).}}
	\begin{itemize}
		\item[\textit{Cause}\hspace{0.27cm}] Raised by \pythoninline{calculate_wua(self)} of \textit{HabitatEvaluation}'s \pythoninline{CHSI()} class in \texttt{HabitatEvaluation/cHSI.py} when a CHSI polygon contains an invalid value.
		\item[\textit{Remedy}] Review cHSI rasters \texttt{HabitatEvaluation/CHSI/\textit{condition}/}.\\
	\end{itemize}

	\item[$\triangleright$]\textbf{\texttt{WARNING: computation{\myUnderscore}extents.xls contains too many reach names.}}
	\begin{itemize}
		\item[\textit{Cause}\hspace{0.27cm}] Raised by \pythoninline{Read().get_reach_info(self, type)} in \texttt{.site{\myUnderscore}packages/riverpy/cTerrainIO.py} when \texttt{ModifyTerrain/.templates/computation{\myUnderscore}extents.xlsx} contains more than eight reach names in columns \texttt{B} and/or \texttt{C}.
		\item[\textit{Remedy}] Ensure that \texttt{ModifyTerrain/.templates/computation{\myUnderscore}extents.xlsx} does not contain more than eight reaches, i.e., only cells \texttt{B6:C13} contain reach names and identifiers (cf. Sec.~\ref{sec:mtsetreaches}).\\
	\end{itemize}	
	
	\item[$\triangleright$]\textbf{\texttt{WARNING: Conversion to polygon failed (FEAT).}}
	\begin{itemize}
		\item[\textit{Cause}\hspace{0.27cm}] Raised by \pythoninline{identify_best_features(self)} in \texttt{MaxLifespan/cActionAssessment.py} when the \pythoninline{arcpy.RasterToPolygon_conversion(FEAT raster)} failed, e.g., because of an empty \pythoninline{FEAT} raster.
		\item[\textit{Remedy}] An empty \pythoninline{FEAT} raster of best lifespans occurs when the feature has no spatial relevance. Consider other terrain modifications or maintenance features to increase the features lifespans and start over planning the feature (set).\\
	\end{itemize}
	
	\item[$\triangleright$] \textbf{\texttt{WARNING: Could not clear/remove .cache.}}
	\begin{itemize}
		\item[\textit{Cause}\hspace{0.27cm}] All modules may raise this warning message when the content in the \texttt{.cache} folder was accessed and locked by another software.
		\item[\textit{Remedy}] Make sure that no other software, including \textit{ArcMap} Desktop or \texttt{explorer.exe} uses the \texttt{MODULE/.cache} folder.\\
	\end{itemize}
	
	\item[$\triangleright$]\textbf{\texttt{WARNING: Could not clean up PDF map temp{\myUnderscore}pages.}}
	\begin{itemize}
		\item[\textit{Cause}\hspace{0.27cm}] The \pythoninline{make_pdf_maps(self, *args)} or \pythoninline{finalize_map(self)} functions of \pythoninline{Mapper()} classes in either \texttt{Life spanDesign/cMapLifespanDesign.py}, \texttt{MaxLifespan/cMapActions.py} or \texttt{Modify}\\\texttt{Terrain/cMapModifiedTerrain.py} create single \texttt{PDF}s of every map image. These single-page \texttt{PDF}s are finally combined into one \texttt{PDF} map assembly and the single-page \texttt{PDF}s are deleted afterward. If the single-page \texttt{PDF}s are locked by another process or corrupted, the \pythoninline{make_pdf_maps(self, *args)} function raises this warning message when it cannot remove temporary .
		\item[\textit{Remedy}] Ensure that no other program is using the \texttt{PDF} files in \texttt{MODULE/Output/Maps/\textit{condition}/} while mapping is in progress.\\
	\end{itemize}
	
	\item[$\triangleright$]\textbf{\texttt{WARNING: Could not clear temp.lyr}}
	\begin{itemize}
		\item[\textit{Cause}\hspace{0.27cm}] The function \pythoninline{prepare_layout(self)} (\texttt{LifespanDesign/cMapLifespanDesign.py}) prints this warning message when it cannot remove the \texttt{temp} layer from the layout template.
		\item[\textit{Remedy}] Ensure that no other program is using the \texttt{.mxd} files (layout), which is used for the map preparation, or the \texttt{.cache} folder.\\
	\end{itemize}
		
	\item[$\triangleright$]\textbf{\texttt{WARNING: Could not divide [...] by [...]"}}
	\begin{itemize}
		\item[\textit{Cause}\hspace{0.27cm}] Raised by \pythoninline{calculate_relative_exceedance(self)} of \textit{HabitatEvaluation}'s \pythoninline{FlowAssessment()} class in \texttt{Habi tatEvaluation/cHSI.py} when the flow duration curve contains invalid data.
		\item[\textit{Remedy}] Ensure the correct setup of the used flow duration curve in \texttt{HabitatEvaluation/FlowDurationCur ves/}. The data types and file structure must correspond to that of the provided template \texttt{flow{\myUnderscore}duration {\myUnderscore}template.xlsx} and all discharge values need to be positive floats. Review Sec.~\ref{sec:hemakehsi} for details.\\
	\end{itemize}
	
	\item[$\triangleright$]\textbf{\texttt{WARNING: Could not get flow depth raster properties. Setting [...]}}
	\begin{itemize}
		\item[\textit{Cause}\hspace{0.27cm}] The \pythoninline{crop_input_raster(self, ...)} function (\texttt{HabitatEvaluation/cHSI.py}) prints this warning message when it cannot read the raster properties from the defined input flow depth raster.
		\item[\textit{Remedy}] Make sure that the defined flow depth raster exists in \texttt{RiverArchitect/01{\myUnderscore}Conditions/\textit{condition}/}.\\
	\end{itemize}
	
	\item[$\triangleright$]\textbf{\texttt{WARNING: Could not get minimum flow depth [...]. Setting h min [...]}}
	\begin{itemize}
		\item[\textit{Cause}\hspace{0.27cm}] The \pythoninline{crop_input_raster(self, ...)} function (\texttt{HabitatEvaluation/cHSI.py}) prints this warning message when it could not read the minimum flow depth from \texttt{Fish.xlsx}. A default value of 0.1 (ft or m) is used to delineate relevant flow regions.
		\item[\textit{Remedy}] Make sure that the defined Fish species / lifestage is assigned a cover value and at least one flow depth value in \texttt{Fish.xlsx} according to the definitions in Sec.~\ref{sec:hemakecovhsi}.\\
	\end{itemize}

	\item[$\triangleright$]\textbf{\texttt{WARNING: Could not reset styles.}}
	\begin{itemize}
		\item[\textit{Cause}\hspace{0.27cm}] Raised by \pythoninline{Write().write_volumes(self, ...)} in \texttt{.site{\myUnderscore}packages/riverpy/cTerrainIO.py} when the \texttt{template} sheet in the output (template) workbook (\texttt{ModifyTerrain/Output/Spreadsheets/\textit{condition}{\myUnderscore}volumes.xlsx} or \texttt{...volume{\myUnderscore}template.xlsx}) is either locked or not accessible.
		\item[\textit{Remedy}] Ensure that no other program uses \texttt{ModifyTerrain/Output/Spreadsheets/\textit{condition}{\myUnderscore}vol umes.xlsx} or \texttt{...volume{\myUnderscore}template.xlsx} and that both workbooks have not been accidentally deleted.\\
	\end{itemize}
	
	\item[$\triangleright$]\textbf{\texttt{WARNING: Could not read project area extents.}}
	\begin{itemize}
		\item[\textit{Cause}\hspace{0.27cm}] Raised by \pythoninline{CAUA().get_extents(self, ...)} in \texttt{/ProjectMaker/cWUA.py} when the function failed to read the project area extents from the \texttt{ProjectArea.shp} shapefile.
		\item[\textit{Remedy}] Ensure that the texttt{ProjectArea.shp} shapefile is correctly created (in particular the \textit{Attributes Table}), according to Sec.~\ref{sec:pminp2}.\\		
	\end{itemize}
	
	\item[$\triangleright$]\textbf{\texttt{WARNING: Could not set project area extents ().}}
	\begin{itemize}
		\item[\textit{Cause}\hspace{0.27cm}] Raised by \pythoninline{CAUA().set_env(self)} in \texttt{/ProjectMaker/cWUA.py} when the function failed to set project area extents.
		\item[\textit{Remedy}] Occurs when the CHSI Raster associated with a certain discharge is empty. Ignore this Warning if the CHSI Raster was correctly identified as being empty, otherwise, revise CHSI Raser creation with the \textit{HabitatEvaluation} module (part~\ref{part:he}).\\		
	\end{itemize}
	
	\item[$\triangleright$]\textbf{\texttt{WARNING: Design map - Could not assign frequency threshold. [...]}}
	\begin{itemize}
		\item[\textit{Cause}\hspace{0.27cm}] Design maps, such as stable grain size, refer to hydraulic data related to a defined return period. If \pythoninline{design_...} functions ) \texttt{LifespanDesign/cLifespanDesignAnalysis.py}) cannot identify a particular \pythoninline{threshold_freq} value, \pythoninline{design_...} functions automatically try to use hydraulic data related to the first entry of \pythoninline{lifespans} (\texttt{Return periods} entry in \texttt{LifespanDesign/.templates/input{\myUnderscore}defini tions.inp}, see Sec.~\ref{sec:inpfile}).
		\item[\textit{Remedy}] -- Assign a float value to the concerned feature in the \texttt{Mobility frequency threshold} row of the \texttt{LifespanDesign/.templates/threshold{\myUnderscore}values.xlsx[thresholds]} spreadsheet (see also Sec.~\ref{sec:modthresh}).\\
			-- Make sure that the defined defined \texttt{Mobility frequency threshold} float is consistent with the defined \texttt{Return periods} in \texttt{LifespanDesign/.templates/input{\myUnderscore}definitions.inp} (see Sec.~\ref{sec:inpfile}).\\
	\end{itemize}
	
	\item[$\triangleright$]\textbf{\texttt{WARNING: Empty design raster [...]}}
	\begin{itemize}
		\item[\textit{Cause}\hspace{0.27cm}] The analyzed feature is not applicable in the defined range.
		\item[\textit{Remedy}] -- If the feature is not intended to be applied anyway, ignore the warning message.\\
							-- If the feature is intended to be applied, manual terrain modifications adapting the feature's threshold values may be necessary.\\
	\end{itemize}
	
	\item[$\triangleright$]\textbf{\texttt{WARNING: Empty lifespan raster [...]}}
	\begin{itemize}
		\item[\textit{Cause}\hspace{0.27cm}] The analyzed feature is not applicable in the defined range.
		\item[\textit{Remedy}] -- If the feature is not intended to be applied anyway, ignore the warning message.\\
							-- If the feature is intended to be applied, manual terrain modifications adapting the feature's threshold values may be necessary.\\
	\end{itemize}
	\item[]
	
	\item[$\triangleright$]\textbf{\texttt{WARNING: Failed to arrange worksheets.}}
	\begin{itemize}
		\item[\textit{Cause}\hspace{0.27cm}] Raised by \pythoninline{Write().write_volumes(self, ...)} in \texttt{.site{\myUnderscore}packages/riverpy/cTerrainIO.py}\\when it could not bring to front the latest copy of the \texttt{template} sheet in the output (template) workbook (\texttt{Modify Terrain/Output/Spreadsheets/\textit{condition}{\myUnderscore}volumes.xlsx} or \texttt{...volume{\myUnderscore}}\\\texttt{template.xlsx}), which contains the calculation results.
		\item[\textit{Remedy}] Trace back earlier error and warning messages. Ensure that no other program uses \texttt{ModifyTerrain/Output/Spreadsheets/\textit{condition}{\myUnderscore}volumes.xlsx} or \texttt{...volume{\myUnderscore}template.xlsx} and that both workbooks have not been accidentally deleted.\\
	\end{itemize}
	
	\item[$\triangleright$]\textbf{\texttt{WARNING: Failed to write unit system to worksheet.}}
	\begin{itemize}
		\item[\textit{Cause}\hspace{0.27cm}] Raised by \pythoninline{Write().write_volumes(self, ...)} in \texttt{.site{\myUnderscore}packages/riverpy/cTerrainIO.py}\\when it could not write volume (numbers) to a copy of the \texttt{template} sheet in the output (template) workbook (\texttt{ModifyTerrain/Output/Spreadsheets/\textit{condition}{\myUnderscore}volumes.xlsx} or \texttt{...volume{\myUnderscore}}\\ \texttt{template.xlsx}).
		\item[\textit{Remedy}] Trace back earlier error and warning messages. Ensure that no other program uses \texttt{ModifyTerrain/Output/Spreadsheets/\textit{condition}{\myUnderscore}volumes.xlsx} or \texttt{...volume{\myUnderscore}template.xlsx} and that both workbooks have not been accidentally deleted.\\
	\end{itemize}	
	
	\item[$\triangleright$]\textbf{\texttt{WARNING: Flow{\myUnderscore}duration[...].xlsx has different lengths of [...]"}}
	\begin{itemize}
		\item[\textit{Cause}\hspace{0.27cm}] Raised by \pythoninline{get_flow_data(self, *args)} of \textit{HabitatEvaluation}'s \pythoninline{FlowAssessment()} class in \texttt{HabitatEval uation/cHSI.py} when the flow duration curve contains invalid data.
		\item[\textit{Remedy}] Ensure that columns \texttt{B} and \texttt{C} of the flow duration curve workbook have the same length (in particular the last value / row must be the same) and check for empty cells.\\
	\end{itemize}
	
	\item[$\triangleright$]\textbf{\texttt{WARNING: Identification failed (FEAT).}}
	\begin{itemize}
		\item[\textit{Cause}\hspace{0.27cm}] Raised by \pythoninline{identify_best_features(self)} in \texttt{MaxLifespan/cActionAssessment.py} when the analyzed feature cannot matched with the internal best lifespan raster.
		\item[\textit{Remedy}] Features with very low lifespan may result in empty rasters. Consider other terrain modifications or maintenance features to increase the features lifespans and start over planning the feature (set).\\
	\end{itemize}
	
	\item[$\triangleright$]\textbf{\texttt{WARNING: Invalid feature names for column headers.}}
	\begin{itemize}
		\item[\textit{Cause}\hspace{0.27cm}] Raised by \pythoninline{Write().write_volumes(self, ...)} in \texttt{.site{\myUnderscore}packages/riverpy/cTerrainIO.py}\\when it could not write feature names to a copy of the \texttt{template} sheet in the output (template) workbook (\texttt{ModifyTerrain/Output/Spreadsheets/\textit{condition}{\myUnderscore}volumes.xlsx} or \texttt{...volume{\myUnderscore}}\\ \texttt{template.xlsx}).
		\item[\textit{Remedy}] -- Ensure that \texttt{ModifyTerrain/.templates/computation{\myUnderscore}extents.xlsx} contains valid reach descriptions (Sec.~\ref{sec:mtsetreaches}).\\
							   -- Ensure that no other program uses \texttt{ModifyTerrain/Output/Spreadsheets/\textit{condition}{\myUnderscore} volumes.xlsx} or \texttt{...volume{\myUnderscore}template.xlsx} and that both workbooks have not been accidentally deleted.\\
	\end{itemize}
	
	\item[$\triangleright$]\textbf{\texttt{WARNING: Invalid type assignment -- setting reach names to IDs.}}
	\begin{itemize}
		\item[\textit{Cause}\hspace{0.27cm}] Raised by \pythoninline{Read().get_reach_info(self, type)} in \texttt{.site{\myUnderscore}packages/riverpy/cTerrainIO.py} when the \pythoninline{type} argument is not \pythoninline{full_name} or \pythoninline{id}. In this case, the \textit{ModifyTerrain} module uses column \texttt{C} in \texttt{ModifyTerrain/.templates/computation{\myUnderscore}extents.xlsx} for reach names and IDs.
		\item[\textit{Remedy}] This warning message only occurs if the GUI application was changed or when the \textit{ModifyTerrain} module is externally called with bad argument order. Review the argument order/assignments in the external call and ensure that the \pythoninline{type} variable is in the \pythoninline{allowed_types = ["full_name", "id"]} list.\\
	\end{itemize}
	
	\item[$\triangleright$]\textbf{\texttt{WARNING: Invalid unit{\myUnderscore}system identifier.}}
	\begin{itemize}
		\item[\textit{Cause}\hspace{0.27cm}]Raised by \pythoninline{ModifyTerrain.__init__()} in \texttt{ModifyTerrain/cModifyTerrain.py} when the unit system identifier is not either \texttt{us} or \texttt{si}. The program will use the default unit system (\texttt{U.S. customary}).
		\item[\textit{Remedy}] This warning message only occurs if the GUI application was changed or when the \textit{ModifyTerrain} module is externally called with bad argument order. Review the argument order/assignments in the external call \pythoninline{var = mt.ModifyTerrain(condition=..., unit_system=..., ...)}.\\
	\end{itemize}
	
	\item[$\triangleright$]\textbf{\texttt{WARNING: Old logfile is locked [...].}}
	\begin{itemize}
		\item[\textit{Cause}\hspace{0.27cm}] Raised by the \pythoninline{logging_start(logfile_name)} function (multiple classes) when the logfiles are locked by another process. The parenthesis \texttt{[...]} indicate the concerned run task.
		\item[\textit{Remedy}] Ensure that the logfiles of the concerned module are not opened in any other process/program.\\
	\end{itemize}
	
	\item[$\triangleright$]\textbf{\texttt{WARNING: Overwriting existing/old ...}}
	\begin{itemize}
		\item[\textit{Cause}\hspace{0.27cm}] The concerned directory already contains an output file of the same name, which is overwritten now. 
		\item[\textit{Remedy}] Ensure to save important layout files in another directory if overwriting is not desired. Cut and paste relevant layouts and maps after every run of Layout Maker or Map Maker to \texttt{LifespanDesign/Products/.../\textit{condition}/} and modify file names.\\
	\end{itemize}
	
	\item[$\triangleright$]\textbf{\texttt{WARNING: Raster / layout identification failed. Using lifespan ...}}
	\begin{itemize}
		\item[\textit{Cause}\hspace{0.27cm}] Warning message from the \pythoninline{choose_ref_layer(self, feature_type)} function (\texttt{LifespanDesign/cMap LifespanDesign.py}) if it cannot determine the raster type, i.e., whether it is a lifespan or a design raster. In this case, the layer symbology of lifespan maps is assign by default, which can cause errors later on.
		\item[\textit{Remedy}] -- Verify the layout templates (\texttt{.mxd}) in \texttt{LifespanDesign/Output/Mapping/.ReferenceLay outs/} for correct layer names, i.e., \pythoninline{"lf_sym"} for lifespan and \pythoninline{"ds_sym"} for design map templates.\\
													 -- Ensure that all layout templates (\texttt{.mxd}) in \texttt{LifespanDesign/Output/Mapping/.Reference Layouts/} names either start with \texttt{lf} or \texttt{ds} for lifespan and design layouts, respectively.\\
	\end{itemize}
	
	\item[$\triangleright$]\textbf{\texttt{WARNING: Volume value assignment failed.}}
	\begin{itemize}
		\item[\textit{Cause}\hspace{0.27cm}] Raised by \pythoninline{Write().write_volumes(self, ...)} in \texttt{.site{\myUnderscore}packages/riverpy/cTerrainIO.py}\\when it could not write volume (numbers) to a copy of the \texttt{template} sheet in the output (template) workbook (\texttt{ModifyTerrain/Output/Spreadsheets/\textit{condition}{\myUnderscore}volumes.xlsx} or \texttt{...volume{\myUnderscore}}\\ \texttt{template.xlsx}).
		\item[\textit{Remedy}] Trace back earlier error and warning messages. Ensure that no other program uses \texttt{ModifyTerrain/Output/Spreadsheets/\textit{condition}{\myUnderscore}volumes.xlsx} or \texttt{...volume{\myUnderscore}template.xlsx} and that both workbooks have not been accidentally deleted.\\
	\end{itemize}	
%	\item[$\triangleright$]\textbf{\texttt{}}
%	\begin{itemize}
%		\item[\textit{Cause}\hspace{0.27cm}]
%		\item[\textit{Remedy}] text\\
%	\end{itemize}
\end{itemize}
