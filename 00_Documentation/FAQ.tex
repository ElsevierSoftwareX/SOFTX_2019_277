\newpage
\part{Frequently Asked Questions (FAQ)} \label{part:faq}
%\section{FAQ}

\begin{itemize}	
	\item[$\triangleright$] \textbf{How can I change map styles?}\\
	Map styles are controlled by settings made in \texttt{.mxd} layout files. Template layout files are stored in different locations for each module and pointers to rasters or shapefiles should be modified in these templates before mapping functions are executed.
	\begin{itemize}
		\item \texttt{LifespanDesign}:\\
		-- Map layout templates are stored in \texttt{/RiverArchitect/LifespanDesign/Output/Mapping/.ReferenceLayouts/}.\\
		-- Mapping functions use the file \texttt{legend.ServerStyle}, which is located in the \texttt{.ReferenceLay outs} folder. Contrary to \texttt{.style} files, the \texttt{.ServerStyle} file is required because \texttt{arcpy}-Python uses \textit{ArcGIS Engine}, rather than \textit{ArcGIS Desktop}. Own \texttt{.style} files can be created using \textit{ArcMap}'s \texttt{Customize > Style Manager}. From the \texttt{Style Manager}, load the \texttt{LifespanDesign} \texttt{legend.style} file from the \texttt{.ReferenceLayouts} folder. Go to \textit{LegendItems} and double-click on \textit{LYR{\myUnderscore}lf{\myUnderscore}style}. The LifespanDesign module's mapping function accounts for font (size) changes made in the \textit{Label Symbol} or \textit{Description Symbol}. For more guidance on creating styles, \href{http://desktop.arcgis.com/en/arcmap/latest/map/styles-and-symbols/creating-new-styles.htm}{click here}. Next, save (or export) the \texttt{.style} file and convert it to a \texttt{.ServerStyle} file using \texttt{MakeServerStyleSet.exe}, which is typically located in \texttt{C:/Program Files (x86)/ArcGIS/Desktop10.x/bin/}. Note that \texttt{MakeSer verStyleSet.exe} and the \texttt{.style} should to be located in the same folder. Finally, rename the new file to \texttt{legend.ServerStyle} and paste it in \texttt{/RiverArchitect/LifespanDesign/ Output/Mapping/.ReferenceLayouts/}.\\
		-- More descriptions in Sec.~\ref{sec:lfrun}.\\
		\item \texttt{MaxLifespan}: see Sec.~\ref{sec:actoutmaps}.
		\item \texttt{ModifyTerrain}: see Sec.~\ref{sec:mtlyt}.
		\item \texttt{HabitatEvaluation}: No mapping function implemented. For mapping \textit{CHSI} rasters, create own \texttt{.mxd} layout files.
		\item \texttt{ProjectMaker}: see Sec.~\ref{sec:pmmaps}.
	\end{itemize}
	
	\item[$\triangleright$] \textbf{What is a \textit{condition}?}\\
	A \textit{condition} refers to a planning state that is typically characterized by a 4-digits year indicator, followed by a layer specifier. \textit{Condition}al Rasters are stored in \texttt{RiverArchitect /01{\myUnderscore}Conditions/}. For more information, refer to Sec.~\ref{sec:input}.

	
%	\item[$\triangleright$] \textbf{Question}\\
%	Description
%	\begin{itemize}
%		\item .
%		\item .\\
%	\end{itemize}

\end{itemize}
